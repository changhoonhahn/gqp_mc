\section{Discussion} \label{sec:discuss}
We demonstrate in this work that we can derive accurate and precise posteriors
on specific galaxy properties using the {\sc PROVABGS} SED modeling. 
Previous works have typically derived ``best-fit'' point estimates of galaxy
properties by maximizing the likelihood rather than sampling the posterior
(\tood{cite}). 
Optimizing the likelihood requires fewer evaluations of the SPS model and thus
take significantly less computational resources to evaluate. 
However, the full Bayesian posteriors  have key advantages over point estimates
of best-fit galaxy properties derived from optimization. 
First, the maximum-likelihood point estimates do not provide uncertainties on
the galaxy properties, which our posteriors demonstrate are significant. 
Some works include uncertainties estimated from the few likelihood
evaluations~\citep[\emph{e.g.}][]{moustakas2013, boquien2019}. 
\todo{discussion of grid based methods and why MCMC is better than them.} 

However, these methods require strong assumptions on the shape of the
likelihood and are inacurrate when there are significant parameter
degeneracies such as the ones in Figure~\ref{fig:posterior}: \emph{e.g.}
between the burst parameters $f_{\rm burst}$ and SFH NMF basis coefficients
$\{\beta_1, \beta_2, \beta_3, \beta_4\}$. 
There are also degeneracies between the ZH and SFH NMF basis coefficients.
These degeneracies can produce a posterior distribution with multiple modes,
which means that different sets of SPS parameters can produce comparable fits
to the observation (Figure~\ref{fig:posterior}). 
Maximum likelihood methods cannot capture any of these complexities in the
posterior distribution~\citep[\emph{e.g.}][]{boquien2019} and lead to overall
less accurate constraints on galaxy properties.  
Futhermore, as we discuss below, the full posterior on galaxy properties allows
us to correct for the effects of model priors on galaxy properties. 
It also opens the door for Bayesian hierarchical approaches.  


\todo{reiterate and discuss advantages of spectra+photometry that emphasizes
why BGS will be awesome}


\begin{figure}
\begin{center}
\includegraphics[width=0.8\textwidth]{figs/sfh_demo.pdf}
    \caption{
        With the {\sc PROVABGS} SPS model, we can infer posteriors on the full
        star formation and metallicity histories from observations. 
        We present the inferred SFH and ZH for arbitrarily chosen star-forming
        galaxy (blue) and quiescent galaxy (orange) from our \lgal~sample.
        The shaded region represent the 64 and 95\% confidence intervals of the
        SFH and ZH posteriors. 
        For comparison, we include the true SFH and ZH (dashed). 
        The inferred SFH and ZH show good agreement with the true values;
        however, similar to the inferred $\avgsfr$ and $\zmw$, SFH and ZH are
        significantly impacted by priors imposed by the SPS model. 
    } \label{fig:sfh_demo}
\end{center}
\end{figure}


%\todo{Beyond the galaxy properties we discuss in Section~\ref{sec:results}, we can also derive SFH and ZH}
In this work, we primarily focus on the following physical properties of
galaxies: $\log M_*$, $\log\avgsfr$, $\log\zmw$, $\tage$, and $\tauism$. 
The {\sc PROVABGS} SPS model, however, can constrain galaxy properties beyond
these properties. 
The SPS model employs nonparametric SFH and ZH prescriptions based on NMF bases
and the model parameters include coefficients for these bases. 
Posteriors on the SPS model parameters, thus, can be used to derive constraints
on the SFH and ZH. 
In Figure~\ref{fig:sfh_demo}, we present the inferred SFH and ZH of two
simulated galaxies from our \lgal~sample: a star-forming galaxy (blue) and a
quiescent galaxy (orange). 
We mark the 68 and 95\% confidence intervals in the shaded regions. 
For comparison, we include the true SFH and ZH from \lgal~(dashed).  
The inferred SFH and ZH is able to generally recover the true histories. 
We emphasize that current SPS models assume constant ZHs that does not vary
over time~\citep{carnall2017, leja2019}. 
Hence inferring ZH over time is one of the key advantages of our {\sc PROVABGS}
SPS model. 
Similar to the inferred $\avgsfr$ and $\zmw$, the SFH and ZH constraints are
also impacted by the priors imposed by our SPS model. 
We charaterize the impact of the prior in detail in
Appendix~\ref{sec:model_priors}.

The most significant limitation of the {\sc PROVABGS} SED modeling in
accurately inferring the true galaxy properties is the prior on galaxy
properties imposed by the model. 
This prior is a major limitation for any SED modeling
methods~\citep{carnall2017, leja2019}. 
It is a consequence of the fact that galaxy properties are \emph{not}
parameters of the SPS model.
For instance, $\avgsfr$ and $\zmw$ are derived by integrating the SFH and ZH
(Eq.~\ref{eq:prop_eqs}), which are parameterized by $\beta_1, \beta_2, \beta_3,
\beta_4$ and $\gamma_1, \gamma_2$. 
Hence, uniform priors on $\beta$s and $\gamma$s (Section~\ref{sec:infer} and
Table~\ref{tab:params}) do not translate into uniform priors on $\avgsfr$ and
$\zmw$.
Other galaxy properties (\emph{e.g.}~$\tage$, SFH, and ZH) likewise have
non-uniform, and undesireable, priors. 

One way to address this issue  is to choose an SED model that does not impose
extreme priors on galaxy properties and characterize the priors in detail so
that final posteriors can be appropriately  interpreted. 
For our {\sc PROVABGS} model, we explicitly chose our SFH prescription so that
the prior on $\log\avgssfr$ extends the range $-12$ to $-9$ dex.
Furthermore, in Appendix~\ref{sec:model_priors} and
Figures~\ref{fig:model_priors} and~\ref{fig:sfh_prior}, we fully characterize
the prior on $\avgssfr$, $\zmw$, $\tage$, SFH, and ZH.
However, we can go beyond solely minimizing the impact of the priors. 
With an estimate of the prior distribution, we can impose maximum-entropy
priors in a specified distribution~\citep{handley2019}. 
From an estimate of the prior distribution on the galaxy properties, we can
derive a new prior on the SPS model parameters that would impose uniform priors
on the galaxy properties. 
In Hahn in preparation, I will demonstrate that with this new prior, we can
infer posteriors on derived galaxy properties without being significantly
impacted by their prior. 
With this prior correction, we will be able to infer even more accurate
posteriors on the physical properties of galaxies with our {\sc PROVABGS} SED
modeling.

In this work, we use mock observations to demonstrate that we can infer
accurate and precise posteriors on certain galaxy properties.
The mock observations were constructed from \lgal~and include photometry and
spectra. 
To generate the spectra, we scale the SED flux to model the fiber aperture
effect --- DESI spectra only include light from a galaxy collected within the
DESI fiber aperture (Section~\ref{sec:spec}. 
We account for this fiber aperture effect in the {\sc PROVABGS} SED model using 
a normalization factor, $f_{\rm fiber}$ (Section~\ref{sec:sps}). 
In observations, however, fiber aperture effects can be wavelength dependent
(\todo{relevant citation}). 
\todo{reference to Marta's paper on aperture effects}. 
A single $f_{\rm fiber}$ factor would not be sufficient for a fiber aperture
effect with a strong wavelength dependence. 
Other SED models use a more wavelength dependent model for flux aperture effect
such as a Chebyschev polynomials~\citep[\emph{e.g.}][]{}. 
\begin{itemize}
    \item caveats: flux calibration --- what was implemented versus what will
    need to be implemented in observations
\end{itemize}


\begin{itemize}
    \item caveats: theoretical assumptions --- isochrones and stellar
        libraries, summary of the appendix
\end{itemize}

\todo{PROVABGS discussion  paragraph} 
\begin{itemize}
\item despite the caveats, this work demonstrates that the galaxy
    properties inferred will be robust and awesome. Paragraph that mentions
    the extension of all the previous science applicatoins 
\item paragraph that discusses the new science applications with the
    posteriors. 
\end{itemize}
