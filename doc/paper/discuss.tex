\section{Discussion} \label{sec:discuss}
We demonstrate in this work that we can derive accurate and precise posteriors
on galaxy properties using our {\sc PROVABGS} SED modeling. 

\todo{posterior over MAP}

\todo{reiterate and discuss advantages of spectra+photometry that emphasizes
why BGS will be awesome}

\todo{Beyond the galaxy properties we discuss in Section~\ref{sec:results}, we can also derive SFH and ZH}
In this work, we focus on the following physical properties of galaxies: 
$\log M_*$, $\log\avgsfr$, $\log\zmw$, $\tage$, and $\tauism$. 
The {\sc PROVABGS} SPS model, however, can constrain galaxy properties beyond
these properties. 
The SPS model employs nonparametric SFH and ZH prescriptions based on NMF bases
and the model parameters include coefficients for these bases. 
Posteriors on the SPS model parameters, thus, can be used to derive constraints
on the SFH and ZH. 
\todo{Figure demonstrating SFH and ZH constraints} 
Of course, like the galaxy properties we present in this work, the SFH and ZH
constraints are also subject to the priors imposed by our SPS model. 


\todo{model priors and preview of how we can correct for it}

\todo{caveat paragraph} 
\begin{itemize}
    \item caveats: flux calibration --- what was implemented versus what will
    need to be implemented in observations
    \item caveats: theoretical assumptions --- isochrones and stellar
        libraries, summary of the appendix
\end{itemize}

\todo{PROVABGS discussion  paragraph} 
\begin{itemize}
\item despite the caveats, this work demonstrates that the galaxy
    properties inferred will be robust and awesome. Paragraph that mentions
    the extension of all the previous science applicatoins 
\item paragraph that discusses the new science applications with the
    posteriors. 
\end{itemize}
