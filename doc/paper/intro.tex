\section{Introduction} \label{sec:intro} 
\todo{What is DESI?} 
Provide an overview of DESI specifics, numbers, and science goals, which will 
mostly be cosmology (BAO, RSD, etc). DESI will be great

In addition to the impact DESI will have on cosmology, DESI will also provide a
transformative data set for galaxy science. \todo{emphasize DESI samples useful
for galaxy science: magnitude complete sample of BGS, LRGs, QSOs}.
With this statistically powerful sample of, 
\todo{Brief list of some exciting galaxy and quasar physics you can do with the
DEIS BGS, QSOs, etc samples.}

\todo{Current status of DESI} 
\begin{itemize}
    \item Comissioning completed
    \item SV data by the end of the year
\end{itemize}
As we prepare for the imminent torrent of data, in this paper we present the
timely mock challenge of the DESI Galaxy Quasar Physics (hereafter GQP) working 
group.  
\todo{what is the mock challenge?} 

\todo{Why do we need a mock challenge?}
We want to test and cement our methodology specifically for our GQP 
analysis before SV data comes out. 
As part of the survey preparation, we have all the tools to accurately 
forward model observations. 
\todo{details on some of the specific tools and what we're able to simulate: 
realistic spectroscopy. realistic photometry. realistic spectro-photometry} 
All of this gives us a rare opportunity to test our methodology on bespoke
simulations. 

A mock challenge is also great for testing new methodology.
BGS is a bright time survey and will push the boundaries of low SNR 
spectra. But if we can find a way to  infer robust galaxy properties the 
statistical payout is awesome. \todo{Something also about LRGs} 
We're also trying to robustly fit spectra and photometry simultaneously. 
This has been done before (\todo{citations}) but not extensively tested 
on simulations. 

\todo{broader impact of mock challenge}
highlight some gqp science cases with a focus on mock challenge papers

\begin{figure}
\begin{center}
\includegraphics[width=\textwidth]{figs/bgs.png} 
\caption{DESI will conduct the largest spectroscopic survey to 
date covering ${\sim}14,000~{\rm deg}^2$. During dark time, DESI will measure
${>}20$ million spectra of luminous red galaxies, emission line galaxies, and 
quasars out to $z > 3$. During bright time, DESI will measure the spectra of 
${\sim}10$ million galaxies out to $z{\sim}0.4$  with the Bright Galaxy Survey (BGS).
{\em Left}: With its ${\sim}14,000~{\rm deg}^2$ footprint (black), DESI will 
cover ${\sim}2\times$ the SDSS footprint (blue; ${\sim}7000~{\rm deg}^2$) 
and ${\sim}45\times$ the GAMA footprint (orange; ${\sim}300~{\rm deg}^2$). 
{\em Right}: Over this footprint, the BGS will provide spectra for a magnitude 
limited sample of ${\sim}10$ million galaxies down to $r < 20$, two orders of 
magnitude deeper than the SDSS main galaxy sample and $0.2$ mag deeper than GAMA.}
\label{fig:bgs}
\end{center}
\end{figure}
