\documentclass[12pt, letterpaper, preprint]{aastex63}
\usepackage[T1]{fontenc}
\usepackage{fontawesome}
\usepackage{color}
\usepackage{amsmath}
\usepackage{natbib}
\usepackage{ctable}
\usepackage{bm}
\usepackage[normalem]{ulem} % Added by MS for \sout -> not required for final version
\usepackage{xspace}
%\usepackage{csvsimple} 

% typesetting shih
\linespread{1.08} % close to 10/13 spacing
\setlength{\parindent}{1.08\baselineskip} % Bringhurst
\setlength{\parskip}{0ex}
\let\oldbibliography\thebibliography % killin' me.
\renewcommand{\thebibliography}[1]{%
  \oldbibliography{#1}%
  \setlength{\itemsep}{0pt}%
  \setlength{\parsep}{0pt}%
  \setlength{\parskip}{0pt}%
  \setlength{\bibsep}{0ex}
  \raggedright
}
\setlength{\footnotesep}{0ex} % seriously?

% citation alias

% math shih
\newcommand{\setof}[1]{\left\{{#1}\right\}}
\newcommand{\given}{\,|\,}

\newcommand{\Om}{\Omega_{\rm m}} 
\newcommand{\Ob}{\Omega_{\rm b}} 
\newcommand{\OL}{\Omega_\Lambda}
\newcommand{\smnu}{M_\nu}
\newcommand{\sig}{\sigma_8} 
\newcommand{\mmin}{M_{\rm min}}
\newcommand{\BOk}{\widehat{B}_0} 
\newcommand{\hmpc}{\,h/\mathrm{Mpc}}
\newcommand{\bfi}[1]{\textbf{\textit{#1}}}
\newcommand{\parti}[1]{\frac{\partial #1}{\partial \theta_i}}
\newcommand{\partj}[1]{\frac{\partial #1}{\partial \theta_j}}
\newcommand{\mpc}{h/{\rm Mpc}}
\newcommand{\lgal}{{\sc LGal}}
\newcommand{\fsps}{{\sc FSPS}}
\newcommand{\tlb}{t_{\rm lookback}}
\newcommand{\zmw}{Z_{\rm MW}}
\newcommand{\tage}{t_{\rm age, MW}}
\newcommand{\tauism}{\tau_{\rm ISM}}
\newcommand{\avgsfr}{\overline{\rm SFR}_{\rm 1Gyr}}
\newcommand{\avgssfr}{\overline{\rm SSFR}_{\rm 1Gyr}}
\let\oldAA\AA
\renewcommand{\AA}{\text{\normalfont\oldAA}}
\newcommand{\github}{\href{https://github.com/changhoonhahn/provabgs/}{\faGithub}}


\makeatletter \def\@captype{figure} \makeatother

\newcommand{\specialcell}[2][c]{%
  \begin{tabular}[#1]{@{}c@{}}#2\end{tabular}}
% text shih
\newcommand{\etal}{\emph{et~al.}}
\newcommand{\bitem}{\begin{itemize}}
\newcommand{\eitem}{\end{itemize}}
%\newcommand{\beq}{\begin{equation}}
%\newcommand{\eeq}{\end{equation}}

%% collaborating
\newcommand{\todo}[1]{\marginpar{\color{red}TODO}{\color{red}#1}}
\definecolor{orange}{rgb}{1,0.5,0}
\newcommand{\ch}[1]{{\color{orange} #1}}

\shorttitle{PROVABGS Mock Challenge}
\shortauthors{Hahn et al.}

\begin{document} \sloppy\sloppypar\frenchspacing 

\title{The DESI PRObabilistic Value-Added Bright Galaxy Survey (PROVABGS) Mock Challenge} 
%\date{\texttt{DRAFT~---~\githash~---~\gitdate~---~NOT READY FOR DISTRIBUTION}}

\newcounter{affilcounter}
\author{ChangHoon Hahn}
\email{changhoon.hahn@princeton.edu}
\affil{Department of Astrophysical Sciences, Princeton University, Peyton Hall, Princeton NJ 08544, USA} 

\author{Gyubin Kwon}
\affil{Astronomy Department, University of California at Berkeley, Berkeley, CA 94720, USA}

\author{Rita Tojeiro}
\affil{School of Physics and Astronomy, University of St Andrews, North Haugh, St Andrews, KY16 9SS, UK}

\author{Malgorzata Siudek} 
\affil{Institut de F\'{\i}sica d'Altes Energies (IFAE), The Barcelona Institute of Science and Technology, 08193 Bellaterra (Barcelona), Spain}
\affil{National Centre for Nuclear Research, ul. Pasteura 7, 02-093, Warsaw, Poland}

\author{Rebecca E. A. Canning}
\affil{}

\author{Mar Mezcua}
\affil{Institute of Space Sciences (ICE, CSIC), Campus UAB, Carrer de Can Magrans, 08193, Barcelona, Spain}
\affil{Institut d’Estudis Espacials de Catalunya (IEEC), C/ Gran Capità, 08034 Barcelona, Spain}

\author{John Moustakas}
\affil{Department of Physics and Astronomy, Siena College, 515 Loudon Road, Loudonville, NY 12211, USA}

\author{Jeremy L. Tinker}
\affil{Center for Cosmology and Particle Physics, Department of Physics, New York University, New York, USA, 10003}

\author{GQP WG}

\begin{abstract}
    The PRObabilistic Value-Added Bright Galaxy Survey (PROVABGS) will provide
    measurements of galaxy properties, such as stellar mass ($M_*$), star
    formation rate (SFR), stellar metallicity ($Z_{\rm MW}$), and stellar age
    ($t_{\rm age, MW}$), for ${>}10$ million galaxies of the DESI Bright Galaxy
    Survey.
    Full posterior distributions of the galaxy properties will be inferred
    using state-of-the-art Bayesian spectral energy distribution (SED) modeling
    of DESI spectroscopy and Legacy Surveys photometry.
    In this work, we present the SED model, Bayesian inference framework, and
    methodology of PROVABGS. %\footnote{publicly available at }.
    Furthermore, we construct realistic synthetic DESI spectra and photometry
    using galaxies in the {\sc L-Galaxies} semi-analytic model
    %, based on their star formation and chemical enrichment histories, 
    and apply the PROVABGS SED modeling on the mock observations.
    We then compare the galaxy properties that we infer to the true galaxy
    properties of the simulation using a hierarchical Bayesian framework
    to quantify accuracy and precision. 
    Overall, we accurately infer the true $M_*$, SFR, $Z_{\rm MW}$, and 
    $t_{\rm age, MW}$ of the simulated galaxies. 
    However, we find that priors on galaxy properties induced by the SED model
    have a significant impact on the posteriors. 
    We characterize the priors and their impact in detail: they impose a 
    ${\rm SFR}{>}10^{-1} M_\odot/{\rm yr}$ lower bound on SFR, a ${\sim}0.3$
    dex bias on $\log Z_{\rm MW}$ for galaxies with low spectral
    signal-to-noise, and $t_{\rm age, MW} < 8\,{\rm Gyr}$ upper bound on
    stellar age. 
    We also demonstrate that a joint analysis of spectra and photometry
    significantly improves the constraints on galaxy properties over photometry
    alone and is necessary to mitigate the impact of the priors. 
    With the methodology presented and validated in this work, PROVABGS will
    maximize information extracted from DESI observations and provide a galaxy
    catalog that will extend current galaxy studies to new regimes and unlock 
    cutting-edge probabilistic analyses. \github
\end{abstract}

\keywords{
    cosmology: observations -- galaxies: evolution -- galaxies: statistics 
}

% --- intro ---  
\section{Introduction} \label{sec:intro} 
The Dark Energy Spectroscopic Instrument will conduct the largest spectroscopic
galaxy survey to date covering ${\sim}14,000~{\rm
deg}^2$~\citep{desicollaboration2016}. 
Over the next five years, 

\todo{What is DESI?} 
Provide an overview of DESI specifics, numbers, and science goals, which will 
mostly be cosmology (BAO, RSD, etc). DESI will be great
Beyond its impact on cosmology, DESI will also be transformative for galaxy
science. 


provide value-added galaxy
catalogs (VAGCs), which will be transformative for galaxy science. 
In the past, VAGCs such as the
MPA-JHU\footnote{https://wwwmpa.mpa-garching.mpg.de/SDSS/DR7/}
provided galaxy properties from emission line analyses of SDSS
spectra~\cite{brinchmann2004}. 

Similarly, the NYU-VAGC~\citep{blanton2005}, 

These catalogs have been indispensible for establishing the global statistical
view of galaxy properties~\citep[see][for a review]{blanton2009}. 
\todo{previous successes with value-added catalogs}
%Increasingly sophisticated statistical studies of the overall population of galaxies as a function of mass, cosmic time, and environment have provided a basic picture of the formation and evolution of galaxies

\todo{what will PROVABGS be good for?} 
extending previous statistcal analyses on SDSS to a larger sample and to higher
redshift. 
Mention some obvious ones: stellar mass function, luminosity function,
star-formation sequence.  
more detailed galaxy-halo connection 
mention some cosmological applications: multi-tracer analyses.  
Furthermore, since BGS will measure galaxy spectra down to $r\sim20$, we will
have faint dwarf galaxies at low redshift. 
provabgs introduces a new frontier of low signal-to-noise statistically
powerful sample that will require  


\todo{Why do we need a mock challenge?}
We want to test and cement our methodology specifically for our GQP 
analysis before SV data comes out. 
As part of the survey preparation, we have all the tools to accurately 
forward model observations. 
\todo{details on some of the specific tools and what we're able to simulate: 
realistic spectroscopy. realistic photometry. realistic spectro-photometry} 
All of this gives us a rare opportunity to test our methodology on bespoke
simulations. 

A mock challenge is also great for testing new methodology.
BGS is a bright time survey and will push the boundaries of low SNR 
spectra. But if we can find a way to  infer robust galaxy properties the 
statistical payout is awesome. \todo{Something also about LRGs} 
We're also trying to robustly fit spectra and photometry simultaneously. 
This has been done before (\todo{citations}) but not extensively tested 
on simulations. 


\begin{figure}
\begin{center}
\includegraphics[width=\textwidth]{figs/bgs.pdf} 
\caption{DESI will conduct the largest spectroscopic survey to 
date covering ${\sim}14,000~{\rm deg}^2$. During dark time, DESI will measure
${>}20$ million spectra of luminous red galaxies, emission line galaxies, and 
quasars out to $z > 3$. During bright time, DESI will measure the spectra of 
${\sim}10$ million galaxies out to $z{\sim}0.4$  with the Bright Galaxy Survey (BGS).
{\em Left}: With its ${\sim}14,000~{\rm deg}^2$ footprint (black), DESI will 
cover ${\sim}2\times$ the SDSS footprint (blue; ${\sim}7000~{\rm deg}^2$) 
and ${\sim}45\times$ the GAMA footprint (orange; ${\sim}300~{\rm deg}^2$). 
{\em Right}: Over this footprint, the BGS will provide spectra for a magnitude 
limited sample of ${\sim}10$ million galaxies down to $r < 20$, two orders of 
magnitude deeper than the SDSS main galaxy sample and $0.2$ mag deeper than GAMA.}
\label{fig:bgs}
\end{center}
\end{figure}

% --- sims ---  
\section{Simulations}\label{sec:sims}
In this Section, we describe how we construct mock observations from simulated
galaxies of the {\sc L-Galaxies} semi-analytic galaxy formation model (SAM).
We use a forward model that includes realistic noise, instrumental effects, and
observational systematics to produce DESI-like photometry and spectra. 
Later, we apply Bayesian SED modeling to these mock observations and
demonstrate that we can accurately infer the true galaxy propertries.

\begin{figure}
\begin{center}
\includegraphics[width=\textwidth]{figs/fm_photo.pdf}
\caption{{\em Left}: We forward model DESI optical $g$, $r$, and $z$ band
    photometry (red) for our simulated galaxies (Section~\ref{sec:lgal}) by
    convolving their SEDs (black dotted) with the broadband filters (dashed)
    and then applying an empirical noise model based on BGS objects in LS
    (Section~\ref{sec:photo}).
    {\em Right}: The $g-r$ and $r-z$ color distribution of the forward modeled
    \lgal~photometry is in good agreement with the color distribution of LS BGS
    objects (black contours). 
    \edits{We only include 425 of the 2,123 total \lgal~mock photometry for clarity.}
    } \label{fig:photo}
\end{center}
\end{figure}

\subsection{L-Galaxies} \label{sec:lgal}
{\sc L-Galaxies}~\citep[hereafter \lgal;][]{henriques2015} is a state-of-the-at
semi-analytic galaxy formation model run on subhalo merger trees from the
Millennium~\citep{springel2005a} and Millenium-II~\citep{boylan-kolchin2009}
$N$-body simulations. 
Millenium-I and II provide a dynamic range of $10^{7.0} < M_* < 10^{12}
M_\odot$ and adopts a \cite{planckcollaboration2014a} $\Lambda$CDM cosmology.
\lgal~includes prescriptions for gas infall and cooling, star formation, disc
and bulge formation, stellar and black hole feedback, and the environmental
effects of tidal and ram-pressure stripping.
Feedback from active galactic nuclei (AGN), which prevents hot gas from
cooling, is the major mechanism for quenching star formation in massive
galaxies.
\lgal~model parameters are calibrated against the observed stellar mass
functions and passive (quiescent) fractions at four different redshifts from 
$z = 3$ to 0.
We refer readers to \cite{henriques2015} for further detail on \lgal.

% A recent comparison to the cosmological hydrodynamical simulation
% IllustrisTNG,  \lgal was recently compared to stellar mass functions and the stellar masses of individual galaxies agree
%  to better than  ∼0.2  dex with IllustrisTNG (Mohammadreza2021)

\subsection{Spectral Energy Distributions} \label{sec:sed}
For each simulated galaxy, \lgal~provides the star formation histories (SFHs)
and chemical enrichment histories (ZH) for its bulge and disk components,
separately, in approximately log-spaced lookback time bins.  
We treat each lookback time bin, $i$, as a single stellar population (SSP) of
age $t_i$.
Then, we derive the luminosities of each component by summing up the
luminosities of their SSPs:
\begin{equation}
    L^{\rm comp.}(\lambda) = \sum \limits_i \left({\rm SFH}^{\rm comp.}_i
    \Delta t_i\right)~L_{\rm SSP}(\lambda\,;\, t_i, Z^{\rm comp.}_i). 
\end{equation}
${\rm SFH}^{\rm comp.}_i$ and $Z^{\rm comp.}_i$ are the star formation rate and
metallicity of the bulge or disk component in lookback time bin $i$. 
$\Delta t_i$ is the width of the bin. 
$L_{\rm SSP}$ corresponds to the luminosity of the SSP, which we calculate
using the Flexible Stellar Population Synthesis~\citep[\fsps;][]{conroy2009,
conroy2010c} model.
% do we want to describe SPS models? 
\edits{
    For \fsps, we use the MIST isochrones~\citep{paxton2011, paxton2013,
    paxton2015, choi2016, dotter2016} and the \cite{chabrier2003} initial mass
    function (IMF). 
    Also, we use the default spectral library in \fsps: the MILES spectral
    library~\citep{sanchez-blazquez2006} over the wavelength range
    $3800-7100\AA$ and the BaSeL
    library~\citep{lejeune1997, lejeune1998, westera2002} 
    outside of those limits.
}

% applying velocity dispersion to bulge and disk components separately 
Next, we apply velocity dispersions to $L^{\rm comp.}(\lambda)$.
For the disk, we apply a fixed $50\,\mathrm{km/s}$ velocity dispersion. 
For the bulge, we derive its velocity dispersion using the~\cite{zahid2016}
empirical relation that depends on the total bulge mass.
Afterwards, we apply dust attenuation to stellar emission in the disk component
($L^{\rm disk}$) based on the cold gas content and orientation of the disk. 
The attenuation curve is derived using a mixed-screen model with the
\cite{mathis1983} dust extinction curve. 
Stellar emission from stars younger than $30{\rm Myr}$ are further attenuated
with a uniform dust screen and a wavelength dependent optical depth.
No dust attenuation is applied to the bulge component.
We use the same dust attenuation that \cite{henriques2015} uses to construct
galaxy colors. 

Finally, we combine the attenuated disk component and the bulge component to
construct the total luminosity of the simulated galaxy and then convert this
rest-frame luminosity to observed-frame SED flux using its redshift, $z$.
\begin{equation}\label{eq:sed} 
    f_{\rm SED}(\lambda) = \frac{A(\lambda)L^{\rm disk}(\lambda) + L^{\rm bulge}(\lambda)}{4 \pi d_L(z)^2 (1+z)}.
\end{equation}
$A(\lambda)$ here is the dust attenuation for the disk component described
above and $d_L(z)$ is the luminosity distance.
In the left panel of Figure~\ref{fig:photo}, we present an example of the SED
flux constructed for an arbitrary \lgal~galaxy (black dotted).

\subsection{Forward Modeling DESI Photometry} \label{sec:photo} 
In this section, we describe how we construct realistic LS-like photometry
from the SEDs of simulated galaxies described in the last section.
First, we convolve the SEDs with the broadband filters of the LS to generate
broadband photometric fluxes: 
\begin{equation} \label{eq:photo}
    f_X = \int f_{\rm SED}(\lambda) R_X(\lambda) {\rm d}\lambda.
\end{equation}
$f_{\rm SED}$ is the galaxy SED (Eq.~\ref{eq:sed}) and $R_X$ is the
transmission curve for filter in the $X$ band. 
We generate photometry for the LS $g$, $r$, and $z$ optical bands.
Next, we apply realistic measurement uncertainties to the derived photometry by
roughly sampling the noise distribution of BGS targets from LS DR9. 
We do this by matching each simulated galaxy to a BGS target with the nearest 
$r$-band magnitude and $g-r$ and $r-z$ color.
The photometric uncertainties ($\sigma_X$) and $r$-band fiber flux ($f_r^{\rm
fiber}$) of the BGS object are then assigned to the simulated galaxy. 
We apply photometric noise by sampling a Gaussian distribution with standard
deviation $\sigma_X$: 
\begin{equation}
    \hat{f}_X = f_X + n_X  \quad {\rm where}~n_X \sim \mathcal{N}(0, \sigma_X).
\end{equation} 
Finally, we impose the target selection criteria of BGS~\citep[][Hahn~\etal~in
prep.]{ruiz-macias2021}.
In the left panel of Figure~\ref{fig:photo}, we overplot the forward
modeled photometry (red) on top of the SED flux (black) for an arbitrary
\lgal~galaxy. 
For reference, we also plot $R_X$ for the $g$, $r$, and $z$ bands of LS in
blue, orange, and green, respectively. 
On the right panel, we compare the $g - r$ versus $r - z$ color distribution
for the forward modeled \lgal~galaxies (red) to the color distribution of BGS
objects in LS (black contour). 
The errorbars represent the photometric uncertainties. 
The forward modeled photometry show good agreement with LS BGS targets in
optical color space.

\begin{figure}
\begin{center}
\includegraphics[width=0.72\textwidth]{figs/fm_spec.pdf}
\caption{
    We construct simulated DESI spectra (solid) for \lgal~simulated galaxies by
    applying a fiber aperture correction to the SED (dashed) and a realistic
    DESI noise model. 
    We apply a fiber aperture correction by scaling down the full SED (dotted)
    by the $r$-band fiber fraction derived from LS imaging. 
    The noise model accounts for the DESI spectrograph response and the bright
    time observing conditions of BGS (\ch{Hahn~\etal~in prep.,
    Schlafly~\etal~in prep.}).  
    We represent the spectra from the three arms ($b$, $r$, and $z$) of the
    DESI spectraphs in blue, orange, and green respectively. 
    Our forward model produces realistic DESI-like spectra that accurately
    reproduce the noise levels and characteristics of actual BGS spectra. 
    } \label{fig:spec}
\end{center}
\end{figure}

\subsection{Forward Modeling DESI Spectra} \label{sec:spec}
In the following, we describe how we construct realistic DESI-like spectroscopy 
from the SEDs of simulated galaxies. 
We begin forward modeling the fiber aperture effect. % and apply a noise model that accurately reproduces the bright time observations of BGS. 
DESI uses fiber-fed spectrographs with fibers that have angular radii of 1''. 
Hence, only the light from a galaxy within this fiber aperture is collected by
the instrument.
LS provides measurements of photometric fiber flux within a 1'' radius aperture
($f_X^{\rm fiber}$), which estimates the flux that passes through to the fibers.
When we assigned photometric uncertainties to our simulated galaxies based on
$r$, $g-r$, and $r-z$ in Section~\ref{sec:photo}, we also assigned $r$-band
fiber flux. 
We model the SED flux that passes through the fiber (fiber loss) by scaling the 
SED flux by the $r$ band fiber fraction, the ratio of $f_r^{\rm fiber}$ over
the total $r$ band flux: 
\begin{equation}
    f^{\rm spec}(\lambda) = \left(\frac{f_r^{\rm fiber}}{f_r}\right)f_{\rm SED}(\lambda).
\end{equation}
This fiber aperture correction assumes that there is no significant color
dependence. 
It also assume that there are no significant biases in the fiber flux
measurements in LS due to miscentering of objects. 
We discuss the implications of these assumptions later in
Section~\ref{sec:discuss}. 
In addition to the aperture correction, we also use $f_r^{\rm fiber}$ to derive
``measured'' $\hat{f}_r^{\rm fiber}$: 
\begin{equation}
    \hat{f}_r^{\rm fiber} = f_r^{\rm fiber} + n^{\rm fiber}_r \quad~{\rm
    where}~n^{\rm fiber}_r \sim \mathcal{N}\left(0, \frac{f_r^{\rm fiber}}{f_r}
    \sigma_r\right).
\end{equation}
After all, when analyzing actual observations, we do not know the true fiber
fraction. 
We later use $\hat{f}_r^{\rm fiber}$ to set the prior on the nuisance parameter
of our SED modeling (Section~\ref{sec:methods}).

Next, we apply a noise model that simulates the DESI instrument response and
bright time observing conditions of BGS. 
We use the same noise model as the spectral 
simulations\footnote{\href{https://specsim.readthedocs.io/en/stable/guide.html}{https://specsim.readthedocs.io}} 
used for the BGS survey design and validation (Hahn~\etal~in prep.). 
We refer readers to \ch{Schlafly~\etal~(in prep.)} for details about the survey
simulations and \ch{Guy~\etal~(in prep.)} for details on the DESI spectroscopic
data reduction pipeline.
Specifically, we use nominal dark time observing conditions with $180s$
exposure time, which accurately reproduce the spectral noise and redshift
success rates of observed BGS spectra in DESI survey validation observations.
In Figure~\ref{fig:spec}, we present the forward modeled BGS spectrum of an
arbitrary \lgal~galaxy (solid). 
We mark the spectrum from each arm of the three DESI spectrographs separately 
(blue, orange, green).
For reference, we include the full SED (dotted) and fiber fraction scaled SED
(dashed) of the galaxy. 


%\todo{conclude by emphasizing the fact that our simulations cover the full expected
%observable space of DESI BGS and therefore if the pipeline works on our mocks,
%then it should work for every expected type of galaxies in the observations} 

 
% --- methods ---  
\newpage
\section{Joint SED modeling of Photometry and Spectra} \label{sec:methods}
\subsection{Stellar Population Synthesis Modeling} \label{sec:sps} 
PROVABGS will provide galaxy properties inferred from joint SED modeling of
DESI photometry and spectra. 
For the SED modeling, we use a state-of-the-art stellar population synthesis
(SPS) model that uses a non-parametric SFH with a starburst, a non-parametric
ZH that varies with time, and a flexible dust attenuation prescription. 

% describe SFH prescription
The form of the SFH is one of the most important factors in the accuracy of an
SPS model.
In general, the form of the SFH requires balancing between being flexible enough
to describe the wide range of SFHs in observations while not being too flexible
that it can describe any SFH at the expense of constraining power.  
If the model SFH is not flexible enough to describe actual SFHs of galaxies,
then unbiased galaxy properties cannot be inferred using the SPS model. 
For instance, most SPS models~(\emph{e.g.} CIGALE,~\citealt{serra2011,
boquien2019}; BAGPIPES,~\citealt{carnall2018}) use parametric SFH such as the
exponentially declining $\tau$-model.
Such functional forms, however, produce biased estimates of galaxy properties
(\emph{e.g.} $M_*$ and SFR) when used to fit mock observations of simulated 
galaxies~\citep{simha2014, pacifici2015, ciesla2017, carnall2018}.
On the other hand, many non-parametric forms of the SFH are overly flexible
and allow unphysical SFHs~\citep{leja2019}, which unncessarily increases 
parameter degeneracies and discards constraining power. 

In our SPS model, we use a non-parametric SFH with two components: one based on
non-negative matrix factorization~\citep[NMF;][]{lee1999,cichocki2009,
fevotte2011} basis functions and a starburst component.
For the first component, SFH is a linear combination of four NMF SFH bases:
\begin{equation} \label{eq:nmf} 
    {\rm SFH}^{\rm NMF} (t, t_{\rm age}) = \sum\limits_{i=1}^{4} \beta_i
    \frac{s_i^{\rm SFH}(t)}{\int\limits_0^{t_{\rm age}} s_i^{\rm SFH}(t) \,
    {\rm d}t}. 
\end{equation} 
$\{s^{\rm SFH}_i\}$ are the NMF basis functions and $\{\beta_i\}$ are the
coefficients. 
The integral in the denominator normalizes the NMF basis functions to unity. 
We constrain $\sum_i \beta_i = 1$, so the total SFH of the component over the
age of the galaxy ($t_{\rm age})$ is normalized to unity.
$\{s^{\rm SFH}_i\}$ are derived from the Illustris cosmological hydrodynamic
simulation~\citep{vogelsberger2014, genel2014, nelson2015}.
We compile, rebin, and smooth the SFHs of Illustris galaxies and then perform
NMF on them to derive $\{s^{\rm SFH}_i\}$. 
We find that 4 components is sufficient to accurately reconstruct the SFHs
from Illustris. 
We present the NMF SFH bases as a function of lookback time in
left panel of Figure~\ref{fig:nmf}.
By using NMF instead of \emph{e.g.} Principal Component Analysis (PCA), we
ensure that all of the SFH bases are non-negative and, thus, physically
meaningful. 
For further details on the derivation of the NMF bases, we refer readers to
Appendix~\ref{sec:nmf}. 
Assuming that the SFHs of Illustris galaxies resemble the SFHs of real
galaxies, our NMF form provides a compact and flexible representation of the
SFHs. 

The NMF basis functions are derived from smooth SFHs, which means that it does
not include any stochasticity. 
However, observations and high resolution zoom-in hydrodyanmical simulations
both find significant stochasticity in galaxy SFHs~\citep{sparre2017,
caplar2019, hahn2019b, iyer2020}. 
To include stochasticity in our SPS model, we include a starburst component
that consists of a SSP. 
Thus, for the total SFH, we use
\begin{equation} \label{eq:sfh}
    {\rm SFH} (t, t_{\rm age}) = (1 - f_{\rm burst})~{\rm SFH}^{\rm NMF} (t,
    t_{\rm age}) + f_{\rm burst}~\delta_{\rm D}(t - t_{\rm burst}).
\end{equation}
$f_{\rm burst}$ is the fraction of total stellar mass formed during the
starburst; $t_{\rm burst}$ is the time at which the starburst occurs; 
$\delta_{\rm D}$ is the Dirac delta function.
In total we use 6 free parameters in our SFH: 4 NMF basis coefficients 
($\beta_i$), $f_{\rm burst}$, and $t_{\rm burst}$. 

% describe ZH 
Another key part of an SPS model is the chemical enrichment history, or ZH. 
Current SPS models mostly assume a flat ZH, constant metallicity over
time~\citep{carnall2019a, leja2019}.
Since galaxies do not have constant metallicities throughout their history,
this assumption can significantly bias the inferred galaxy
properties~\citep{thorne2021}. 
Instead, we take a similar approach to the SFH and use NMF basis functions for
ZH:
\begin{equation}
    {\rm ZH}(t) = \sum\limits_{i=1}^2 \gamma_i s_i^{\rm ZH}(t).
\end{equation} 
$\{s_i^{\rm ZH}(t)\}$ are the ZH NMF basis functions and $\{\gamma_i\}$ are the
coefficients. 
$\{s_i^{\rm ZH}(t)\}$ are fit using the ZHs of simulated galaxies from
Illustris in the same fashion as the SFH. 
In the right panel of Figure~\ref{fig:nmf}, we present the ZH NMF bases as a
function of lookback time. 
We use two NMF components, so our ZH prescription has 2 free parameters. 

\begin{figure}
\begin{center}
\includegraphics[width=0.85\textwidth]{figs/nmf_bases.pdf} 
    \caption{
        Non-negative matrix factorization basis functions for the SFH (left)
        and ZH (right) used in the non-parametric SFH and ZH prescriptions of
        our SPS model. 
        These basis functions are derived from the SFHs and ZHs of simulated
        galaxies in the Illustris cosmological hydrodynamic simulations. 
        With the NMF basis functions, we can reproduce the wide range of SFHs
        and ZHs of Illustris galaxies (Appendix~\ref{sec:nmf}).  
    }
    \label{fig:nmf}
\end{center}
\end{figure}

We use the SFH and ZH above to model the unattenuated rest-frame luminosity as
a linear combination of multiple SSPs, evaluated at logarithmically-spaced
lookback time bins.
We use a fixed log-binning with the bin egdes starting with $(0, 10^{6.05}{\rm
yr})$, $(10^{6.05}, 10^{6.15}{\rm yr})$, and continuing on with bins of width
0.1 dex.
The binning is truncated at the age of the model galaxy. 
For a $z=0$ galaxy, this binning produces 43 $\tlb$ bins.
We use log-spaced $\tlb$ bins because it better reproduces galaxy luminosities
evaluated with much higher resolution $\tlb$ binning than linearly-spacing, for
the same number of bins. 
At each of the 43 $\tlb$ bin $i$, we evaluate the luminosity of a SSP with
${\rm ZH}(t_i)$, where $t_i$ is the center of $\tlb$ bin, and total stellar
mass calculated by resampling the SFH in Eq.~\ref{eq:sfh}. 
We use \fsps~to evaluate the SSP luminosities and use the MIST isochrones, the
combination of MILES and BaSeL spectral libraries, and the \cite{chabrier2003}
IMF (same as in Section~\ref{sec:sed}).  
Since we use MIST isochrones, we impose a minimum and maximum limit to ${\rm
ZH}$ based on its coverage: $4.49\times10^{-5}$ and $4.49\times10^{-2}$,
respectively.
These metallicity values are in units of absolute metallicity and can be
converted to solar metallicity using $Z_\odot = 0.019$. 
We note that our stellar metallicity range is significantly broader than
previous studies for additional flexibility~\citep[\emph{e.g.}][]{leja2017,
carnall2019a, tacchella2021}. 
Since we model galaxies solely as a linear combination of SSPs, we do not
model nebular emission.  
We, therefore, exclude emission lines in our SED modeling by masking the
wavelength ranges of emission lines.

Before we combine the SSP luminosities, we apply dust attenuation.
We use a two component \cite{charlot2000} dust attenuation model with birth
cloud (BC) and diffuse-dust (ISM) components. 
The BC component represents the extra dust attenuation of young stars that are
embedded in modecular clouds and HII regions. 
For SSPs younger than $t_i < 100{\rm Myr}$, we apply the
following BC dust attenuation: 
\begin{equation}
    L_i(\lambda) = L_i^{\rm unatten.}(\lambda) \exp\left[-\tau_{\rm BC} \left(
    \frac{\lambda}{5500\AA} \right)^{-0.7} \right].
\end{equation}
$\tau_{\rm BC}$ is the BC optical depth that determines the strength of the BC
attenuation. 
Afterwards, {\em all} SSPs are attenuated by the diffuse dust using the
\cite{kriek2013} attenuation curve parameterization: 
\begin{equation}
    L_i(\lambda) = L_i^{\rm unatten.}(\lambda) \exp\left[-\tau_{\rm ISM} \left(
    \frac{\lambda}{5500\AA} \right)^{n_{\rm dust}} \left(k_{\rm Cal}(\lambda) +
    D(\lambda) \right) \right].
\end{equation}
$\tau_{\rm ISM}$ is the diffuse dust optical depth.
$n_{\rm dust}$ is the \cite{calzetti2001} dust index, which determines the
slope of the attenuation curve. 
$k_{\rm Cal}(\lambda)$ is the \cite{calzetti2001} attenuation curve and
$D(\lambda)$ is the UV dust bump, parameterized using a Lorentzian-like Drude 
profile:
\begin{equation}
    D(\lambda) = \frac{E_b(\lambda~\Delta \lambda)^2}{(\lambda^2 -
    \lambda_0^2)^2 + (\lambda~\Delta \lambda)^2}
\end{equation}
where $\lambda_0 = 2175 \AA$, $\Delta \lambda = 350\AA$, and 
$E_b=0.85 - 1.9\,n_{\rm dust}$ are the central wavelength, full width at half
maximum, and strength of the bump, respectively. 
Once dust attenuation is applied to the SSPs, we sum them up to get the
rest-frame luminosity of the galaxy. 
In total, our SPS model has 12 free parameters: $M_*$, 4 SFH basis
coefficients, $f_{\rm burst}$, $t_{\rm burst}$, 2 ZH basis coefficients,
$\tau_{\rm BC}$, $\tau_{\rm ISM}$, and $n_{\rm dust}$. 

%description of our speculator SED model \citep{alsing2019}, which is based on FSPS. We use Chabrier IMF \ch{do we need to justify htis?}. 

In practice, each model evaulation using \fsps~requires ${\sim}340$ ms. 
%evaluating each SSP using \fsps~requires \ch{X} seconds.  For each model evaluation, we evaluate $\sim 43$ SSPs in each of the log-spaced $\tlb$ bins. 
Though this is not a prohibitive computational cost on its own, sampling a
high dimensional parameter space for inference requires $>100,000$ evaluations
--- \emph{i.e.} $\gtrsim10$ CPU hours \emph{per galaxy}. 
For the >10 million BGS galaxies, this would require >100 million CPU hours. 
Instead, we use an emulator for the model luminosity, which uses a PCA neural
network (NN) following the approach of \cite{alsing2019}. 

%The NN provides a flexible and accurate mapping between the SPS model
%parameters and PCA coefficients --- \emph{i.e.} the NN predicts PCA
%coefficients for a given set of SPS parameters. 
%Then the linear combination of the predicted coefficients and PCA basis
%functions give us the emulated model luminosity. 
%The PCA basis functions and NN are trained using 1,000,000 SPS parameters and
%model luminosity pairs, $\{(\theta, L(\lambda;\theta))\}$. 

To construct our emulator, we first generate $N_{\rm model} = 1,000,000$
model luminosities, $L(\lambda;\theta)$, from unique SPS parameters,
$\theta$, sampled from the prior (Section~\ref{sec:infer},
Table~\ref{tab:params}).
We then split the model luminosities into four wavelength bins: 2000 - 3600,
3600 - 5500, 5500 - 7410, and 7410 - $60000\AA$ with $N_{\rm spec}$ = 127, 2109,
2113, and 549 resolution elements, respectively.
For each wavelength bin, a PCA is done in the $N_{\rm spec}$-dimensional
space to yield PCA basis functions, or eigenspectra. 
We represent the model luminosity using the first $N_{\rm basis}$ = 50, 50, 50, and 30
eigenspectra and their corresponding PCA coefficients. 
A NN is then trained on the set of $N_{\rm model}$ models to
derive a mapping from the 12 SPS parameters to the $N_{\rm basis}$ PCA
coefficients for each wavelength bin. 


Once trained, our emulator works as follows.
For a given set of SPS parameters, the NN for each wavelength bin predicts
PCA coefficients. 
The coefficients are then linearly combined with the eigenspectra to
predict the model luminosity in the wavelength bin. 
The luminosity in all four wavelength bins are concatenated to produce the
full model luminosity. 
Throughout the wavelength range relevant for BGS, $3000 < \lambda < 9800\AA$,
we achieve $< 1\%$ accurate with the emulator. 
For details on the training, validation, and performance of our PCA NN
emulator, we refer readers to Kwon \etal~(in prep.). 
With the neural emulator, each model evaluation only requires 
${\sim}2.9$ ms --- 100$\times$ faster than with FSPS.

From the rest-frame luminosity, we obtain the observed-frame, redshifted, flux
in the same way as Eq.~\ref{eq:sed}.
In our case, redshift is not a free parameter since we will have high quality
spectroscopic redshifts for every DESI BGS galaxy.
BGS redshifts will have small redshift error, $\sigma_z < 0.0005 (1+z)$
(150 km/s), and <5\% catastrophic failures, $\Delta z/(1+z) < 0.003$ (<1000
km/s).
To model DESI photometry, we convolve the model flux with the LS broadband
filters as in Eq.~\ref{eq:photo}.
To model DESI spectra, we first apply Gaussian velocity dispersion. 
In this work, we keep velocity dispersion fixed at 0 km/s as a conservative
test for our SED modeling when we use an explicitly incorrect velocity
dispersion.
Later when we apply our SPS model to observations, the velocity dispersion will
be set to a more realistic value. 
It can also be set as a free parameter.
%In practice, however, the velocity dispersion can be set as a free parameter. 
After velocity dispersions, the broadened flux is resampled into the DESI
wavelength binning.  
Since DESI spectra do not necessarily include all the light of a galaxy, we
include a nuisance parameter $f_{\rm fiber}$, a normalization factor on the
spectra to account for fiber aperture effects. 
Next, the model photometry and spectrum can be directly compared to
observations.

\begin{table} 
\caption{Parameters of the PROVABGS SPS model and their priors used for joint
    SED modeling of DESI photometry and spectroscopy.} 
\begin{center}
    \begin{tabular}{ccc} \toprule
        name & description & prior \\[3pt]
        \hline 
        $\log M_*$                              & log galaxy stellar mass & uniform over [7, 12.5] \\
        $\beta_1, \beta_2, \beta_3, \beta_4$    & NMF basis coefficients for SFH & Dirichlet prior \\
        $f_{\rm burst}$ & fraction of total stellar mass formed in starburst event & uniform over [0, 1] \\
        $t_{\rm burst}$ & time of starburst event & uniform over [10Myr, 13.2Gyr] \\
        $\gamma_1, \gamma_2$ & NMF basis coefficients for ZH & log uniform over
        [$4.5\times10^{-5}, 1.5\times10^{-2}$] \\
        $\tau_{\rm BC}$ & Birth cloud optical depth & uniform over [$0, 3$] \\
        $\tau_{\rm ISM}$ & diffuse-dust optical depth & uniform over [$0, 3$] \\
        $n_{\rm dust}$ & \cite{calzetti2001} dust index & unifrom over[$-2, 1$]\\
        $f_{\rm fiber}$ & spectrum fiber-aperture effect normalization &
        Gaussian $\mathcal{N}(\hat{f}^{\rm fiber}_r, \frac{f^{\rm fiber}_r}{f_r} \sigma_r)$\\
        \hline            
\end{tabular} \label{tab:params}
\end{center}
\end{table}

\begin{figure}
\begin{center}
    \includegraphics[width=0.9\textwidth]{figs/mcmc_posterior_demo.pdf}
    \caption{
        \emph{Top}: 
        Posterior probability distribution of our 12 SPS model parameters
        derived from joint SED modeling of the mock DESI photometry and
        spectrum.
        The contours mark the 68 and 95\% percentiles.
        We use a Gaussian likelihood and the prior specified in
        Table~\ref{tab:params} to evaluate the posterior and sample the
        distribution using ensemble slice MCMC. 
        \emph{With our Bayesian SED modeling approach, we accurately quantify
        uncertainties and capture complexities (\emph{e.g.}~parameter
        degeneracies and multimodality) in the posterior distribution.}\\
        \emph{Bottom}: 
        We compare the best-fit model observables (orange) to the mock
        observations (black).  
        We find excellent agreement for both the LS photometry (left) and the
        DESI spectrum (right). 
    } \label{fig:posterior}
\end{center}
\end{figure}


\subsection{Bayesian Parameter Inference} \label{sec:infer} 
Using the SPS model above, we perform Bayesian parameter inference to derive
posterior probability distributions of the SPS parameters from photometry and
spectroscopy. 
From Bayes rule, we write down the posterior as
\begin{equation} \label{eq:bayes}
    p(\theta\given {\bf X}) \propto p(\theta)~p({\bf X} \given \theta)
\end{equation}
where ${\bf X}$ is the photometry or spectrum and $\theta$ is the set of SPS
parameters. 
$p({\bf X} \given \theta)$ is the likelihood, which we calculate independently for
the photometry
\begin{equation}
    \mathcal{L}^{\rm photo} \propto \exp\left[-\frac{1}{2} \left(\frac{X^{\rm photo} -
    m^{\rm photo}(\theta)}{\sigma^{\rm photo}}\right)\right]
\end{equation}
and for the spectrum
\begin{equation}
    \mathcal{L}^{\rm spec} \propto \exp\left[-\frac{1}{2} \left(\frac{X^{\rm spec} -
    m^{\rm spec}(\theta)}{\sigma^{\rm spec}} \right)^2\right].
\end{equation}
$m^{\rm photo}$ and $m^{\rm spec}$ represent SPS model photometry and spectroscopy. 
$\sigma^{\rm photo}$ and $\sigma^{\rm spec}$ respresent the uncertainties on
the measured photometry and spectrum. 
In calculating $\mathcal{L}^{\rm spec}$, we exclude wavelength ranges of width
40\AA~surrounding the OII, H$\beta$, OIII, and H$\alpha$ emission lines since
our SED model does not model gas emissions.
We consider the photometry indepedent from the spectrum so we combine the
likelihoods when jointly modeling the spectrophotometry: 
\begin{equation}
    \log \mathcal{L} \approx \log \mathcal{L}^{\rm photo} + \log
    \mathcal{L}^{\rm spec}.
\end{equation}
$p(\theta)$ in Eq.~\ref{eq:bayes} is the prior on the SPS parameters. 
For most of our parameters, we use uninformative uniform priors with
conservatively chosen ranges that are listed in Table~\ref{tab:params}. 
However, for the priors of $\{\beta_1, \beta_2, \beta_3, \beta_4 \}$, the NMF coefficients
for the SFH, we use a Dirichlet distribution to maintain the normalization of
the SFH in Eq.~\ref{eq:nmf}. 
With Dirichlet priors, $\beta_i$ are within $0 < \beta_i < 1$ and
satisfy the constraint $\sum_i \beta_i = 1$. 

Now that we can evaluate the posterior at given $\theta$, we estimate the
posterior distributions using Markov Chain Monte Carlo (MCMC) sampling. 
We use the \cite{karamanis2020} ensemble slice sampling algorithm with the
{\sc zeus} Python
package\footnote{\href{https://zeus-mcmc.readthedocs.io/}{https://zeus-mcmc.readthedocs.io/}}. 
Ensemble slice sampling is an extension of standard slice sampling that does
not requires specifying the initial length scale or any further hand-tuning.
It generally converges faster than other MCMC algorithms (\emph{e.g.}
Metropolis) and generates chains with significantly lower autocorrelation.

When we sample the posterior, we do not directly sample our 12 dimensional
SPS parameter space because we use a Dirichlet prior on the SFH NMF
coefficients. 
Dirichlet distributions are difficult to directly sample so we instead use the
\cite{betancourt2012} sampling method, which transforms an $N$ dimensional
Dirichlet distribution into an easier to sample $N-1$ dimensional space.
Hence, we sample the posterior in the transformed 11 dimensional space. 
Given this dimensionality, we run our MCMC sampling with 30 walkers.
Overall, we find that the sampling converges after 2,500 iterations with a 500
iteration burn in. 
Deriving the posterior distribution from a joint SED modeling of photometry and
spectra, with the emulator, takes ${\sim}10$ CPU minutes per galaxy.
In principle, since our emulator uses a PCA NN, we can further expedite our
paremeter inference using more efficient sampling methods that exploit gradient
information, such as Hamiltonian Monte Carlo.  
We will explore further speed ups to our SED modeling in future works. 

In Figure~\ref{fig:posterior} we present the posterior distribution of our 12
SPS model parameters for an arbitrarily chosen \lgal~mock observation. 
We mark the 68 and 95 percentiles of the distribution with the contours. 
The posterior distribution reveal there are significant degeneracies between
SPS parameters: \emph{e.g.} $\beta_2^{\rm SFH}$ and $f_{\rm burst}$. 
Furthermore, the distribution is multimodal (see $f_{\rm burst}$ panels). 
With our Bayesian SED modeling, we are able to capture such complexities in the
posterior that would be lost with point estimates or maximum likelihood
approaches.
In the bottom panels, we compare our SPS model evaluated at the best-fit
parameters (orange) with the \lgal~mock observations (black). 
On the left, we compare the $g$, $r$, $z$ band magnitudes; on the right, we
compare spectra. 
We find excellent agreement between the best-fit SPS model and mock
observations.
The entire PROVABGS SED modeling pipeline, including the neural emulators and
parameter inference framework, is publicly available at
\href{https://github.com/changhoonhahn/provabgs/}{https://github.com/changhoonhahn/provabgs/}. 

% --- results ---  
\section{Results} \label{sec:results}
%\subsection{Inferred Galaxy Properties}
Figuree~\ref{fig:prop_inf} 

\begin{figure}
\begin{center}
\includegraphics[width=\textwidth]{figs/mini_mocha_sfr_100myr_comparison.pdf} 
\caption{The properties inferred from ifsps spetrophotometry fit as a function of true properties. 
}
\label{fig:prop_inf}
\end{center}
\end{figure}

In order to quantify the precision and accuracy of the inferred physical
properties for our simulated galaxy population, we begin by assuming that the
discrepancy between the inferred and true parameters for each galaxy 
$\Delta_{\theta,i}$) 
\begin{equation}
    \theta^{\rm inf}_i = \theta^{\rm true}_i + \Delta_{\theta,i}
\end{equation}
where $\Delta_{\theta,i}$ is sampled from a Gaussian distribution
\begin{equation}
    \Delta_{\theta,i} \sim \mathcal{N}(\mu_{\Delta_{\theta}}, \sigma_{\Delta_{\theta}}).
\end{equation}
This Gaussian distribution is described by population hyperparameters $\mu_{\Delta_{\theta}}$ and 
$\sigma_{\Delta_{\theta}}$, the mean and standard deviation, which quantify the accuracy and 
precision of the inferred physical properties for the population. 

Given the photomety and spectrum of our galaxies, $\{{\bfi D}_i\}$, we can get the posteriors
for these population parameters $\theta_\Delta = (\mu_{\Delta_{\theta}}, \sigma_{\Delta_{\theta}})$ 
using a hierarchical Bayesian framework~\citep{hogg2010a}: 
\begin{align}
p(\theta_\Delta \given \{{\bfi D_i}\}) 
    =&~\frac{p(\theta_\Delta)~p( \{{\bfi D_i}\} \given \theta_{\Delta})}{p(\{{\bfi D_i}\})}\\
    =&~\frac{p(\theta_\Delta)}{p(\{{\bfi D_i}\})}\int p(\{{\bfi D_i}\} \given \{\theta_i\})~p(\{\theta_i\} \given \theta_\Delta)~{\rm d}\{\theta_i\}.
\end{align} 
Naively the posteriors for each of the galaxies are not correlated, so we can factorize the expression above
\begin{align}
p(\theta_\Delta \given \{{\bfi D_i}\}) 
    =&~\frac{p(\theta_\Delta)}{p(\{{\bfi D_i}\})}\prod\limits_{i=1}^N\int p({\bfi D_i} \given \theta_i)~p(\theta_i \given \theta_\Delta)~{\rm d}\theta_i\\
    =&~\frac{p(\theta_\Delta)}{p(\{{\bfi D_i}\})}\prod\limits_{i=1}^N\int \frac{p(\theta_i \given {\bfi D_i})~p({\bfi D_i})}{p(\theta_i)}~p(\theta_i \given \theta_\Delta)~{\rm d}\theta_i\\
    =&~p(\theta_\Delta)\prod\limits_{i=1}^N\int \frac{p(\theta_i \given {\bfi D_i})~p(\theta_i \given \theta_\Delta)}{p(\theta_i)}~{\rm d}\theta_i.
\end{align} 
$p(\theta_i \given {\bfi D_i})$ is the posterior for galaxy $i$. Hence, the
integral can be which means the integral can be estimated using the MCMC sample
from the posterior
\begin{align}
p(\theta_\Delta \given \{{\bfi D_i}\}) 
    =&~p(\theta_\Delta)\prod\limits_{i=1}^N\frac{1}{S_i}\sum\limits_{j=1}^{S_i}
    \frac{p(\theta_{i,j} \given \theta_\Delta)}{p(\theta_{i,j})}.
\end{align} 
$S_i$ is the number of MCMC samples and $\theta_{i,j}$ is the $j^{\rm th}$
sample of galaxy $i$. We present the maximum a posteriori (MAP) estimates of
$\theta_\Delta$ for $\log~M_*$ and $\log~{\rm SFR}$ in
Figure~\ref{fig:specphoto}. 


$\theta_\Delta$ as a function of SNR/mag/colour. 

$\theta_\Delta$ as a function of SNR/mag/colour. 

\begin{figure}
\begin{center}
\includegraphics[width=0.75\textwidth]{figs/photo_vs_specphoto_ifsps_sfr_100myr_vanilla_noise_legacy_bgs0_legacy.pdf} 
\caption{The discrepancies between the inferred and input/``true'' $M_*$s (left) and SFRs 
(right) for our {\sc LGal} galaxies. In blue, we infer $M_*$s and SFRs using only photometry;
in orange, we infer $M_*$s and SFRs by jointly fitting both photometry and spectroscopy. 
{\em Jointly fitting spectroscopy and photometry improves constraints on galaxy properties.}
}
\label{fig:specphoto}
\end{center}
\end{figure}

%\begin{table}
%\caption{$\theta_\Delta$ $\theta_{\rm inf}$ - $\theta_{\rm true}$ and uncertainties for different sets of data fitted with ifsps} 
%\begin{center} 
%\begin{tabular}{ccccc} \toprule
%set & photometry & spectroscopy & specphot \\
%$\Delta M_{tot}$ & 0.13 & 0.11 & 0.09\\
%$M_{err}$ & 0.10 & 0.08 & 0.07 \\
%$\Delta$ Age & 4.05 & 3.84 & 4.33\\
%$Age_{err}$ & 1.83 & 2.37 & 2.03\\
%$\Delta$ Z & 0.0546 & 0.0126 & 0.0063 \\
%$Z_{err}$ & 0.0291 & 0.0203 & 0.0050\\
%\hline 
%\hline            
%\end{tabular} \label{tab:setups}
%\end{center}
%\end{table}

%\begin{figure}
%\begin{center}
%\includegraphics[width=\textwidth]{figs/mini_mocha_cigale_noise_CIGALEA.png} 
%\includegraphics[width=\textwidth]{figs/mini_mocha_cigale_noise_CIGALEB.png} 
%\includegraphics[width=\textwidth]{figs/mini_mocha_cigale_noise_CIGALEC.png}
%\includegraphics[width=\textwidth]{figs/mini_mocha_cigale_noise_CIGALED.png}
%\caption{The properties inferred from CIGALE photometry fit as a function of true properties. Configuration CIGALE A, B, C, and D 
%}
%\label{fig:photo_cigale}
%\end{center}
%\end{figure}

%\begin{figure}
%\begin{center}
%\includegraphics[width=\textwidth]{figs/mini_mocha_ifsps_specphotofit_vanilla_noise_bgs0_legacy_delta.pdf} 
%\caption{delta(galaxy properties) as a function of $M_{tot}$, r mag and colors for ifsps(spectrophotometry) and CIGALE (photometry, CIGALE D). 
%}
%\label{fig:photo_cigaleALL}
%\end{center}
%\end{figure}

{\bf figure}: $\theta_{\rm inf}$ vs $\theta_{\rm true}$ plot for final fitting methods. Money plot of the mock challenge 

{\bf figure}: delta(galaxy properties) as a function of r-mag, SNR(?), color for different fitters for spectro-photometric fitting 

{\bf figure}: $\theta_{\rm inf}$ vs $\theta_{\rm true}$ plot for multiple fitting methods 


% --- discussion ---  
\section{Discussion} \label{sec:discuss}
We demonstrate in this work that we can derive accurate and precise posteriors
on galaxy properties using our {\sc PROVABGS} SED modeling. 


\begin{itemize}
    \item posterior over MAP
    \item reiterate and discuss advantages of spectra+photometry that
        emphasizes why BGS will be awesome
    \item Beyond the galaxy properties we discuss in Section~\ref{sec:results},
        we can also derive SFH and ZH
    \item model priors and preview of how we can correct for it
    \item caveats: flux calibration --- what was implemented versus what will
        need to be implemented in observations
    \item caveats: theoretical assumptions --- isochrones and stellar
        libraries, summary of the appendix
    \item despite the caveats, this work demonstrates that the galaxy
        properties inferred will be robust and awesome. Paragraph that mentions
        the extension of all the previous science applicatoins 
    \item paragraph that discusses the new science applications with the
        posteriors. 
\end{itemize}


% --- summary ---  
\section{Summary}
Over the next five years, DESI will measure spectra for ${>}30$ million
galaxies, including ${>}10$ million galaxies in BGS during bright time.
Each DESI galaxy will also have optical photometry from the Legacy Survey. 
BGS, which will extend out to $z\sim0.6$, will provide a $r < 19.5$
magnitude-limited sample of ${\sim}10$ galaxies spanning a wide range of galaxy
properties with high completeness and ${>}95\%$ redshift efficiency. 
It will also include a sample of fainter galaxies down to $r < 20.175$ selected
based on a fiber magnitude and color. 
This upcoming dataset offers a unique opportunity to leverage its statistical
power for galaxy evolution and maximize its scientific impact. 
For instance, having measured galaxy properties for such a statistically
powerful sample of galaxies would enable us to measure population statistics
and empirical relations of galxaies with unprecedented precision. 
It would also enable more complete and precise comparisons between observations
and galaxy formation models, which will shed light into the physical processes
of galaxy evolution.
To exploit this opportunity, we will construct the PRObabilistic Value-Added
Bright Galaxy Survey (PROVABGS), where we will apply state-of-the-art Bayesian
SED modeling to jointly analyze DESI spectroscopy and LS photometry. 
PROVBGS will provide full posterior distributions of galaxy properties, such as
stellar mass ($M_*$), star formation rate (SFR), stellar metallicity 
($Z_{\rm MW}$), and stellar age ($\tage$), for ${>}10$ million BGS galaxies.

In this work, we present and validate the SED model, Bayesian inference
framework, and other methods that will be used to construct PROVABGS.
We use \todo{2239} galaxies in the {\sc L-Galaxies} semi-analytic model to
construct realistic synthetic DESI spectra and photometry.  
We build SEDs using stellar population synthesis based on the star formation
and chemical enrichment histories of the simulated galaxies.
Then we simulate the SEDs using the forward modeling pipeline used in the BGS
survey design.  
Afterwards, we apply the PROVABGS SED modeling on the mock DESI observations to
derive posteriors on $M_*$, $\avgsfr$, $\zmw$, and $\tage$. 
From the posteriors and the population inference we conduct to quantify the
accuracy and precision, we find: 
\begin{itemize}
    \item Overall we derive posteriors on galaxy properites that are in good
        agreement with the true properties of the simulated galaxies. 
        Furthermore, with posteriors rather than point estimates we accurately
        estimate the uncertainties on the galaxy properties. 
        We infer posteriors with the following levels of precision: 
        $\sigma_{M_*}\sim0.1$dex, $\sigma_{\log\avgsfr}\sim0.1$dex, 
        $\sigma_{\log\zmw}\sim0.15$dex, and $\sigma_{\tage}\sim0.5$ Gyr. 
        Our results also demonstrate that we successfully marginalize over the
        effect of dust and other nuisance parameters. 
    \item Like any SED model, the PROVABGS SED model also imposes significantly
        non-uniform priors on galaxy properties. 
        We find that these priors impose a lower bound on $\avgsfr$ of 
        $\avgsfr > 10^{-1}M_\odot/{\rm yr}$. 
        It also biases $\zmw$ by ${\sim}0.3$dex for observations with low
        spectral signal-to-noise and imposes an upper bound of $\tage < 8$ Gyr
        on $\tage$. 
        We characterize the priors in detail so that constraints on galaxy
        properties can be interpreted in future use of PROVABGS. 
    \item We compare the posteriors derived from DESI spectrophotometry to
        those derived from photometry alone. 
        Including DESI spectra substantially improves the constraints on galaxy
        properties. 
        In fact, jointly analyzing spectra is {\em essential} for mitigating
        the impact of the SED model priors. 
        For example, with photometry alone, the priors impose a more
        restrictive $\avgsfr > 1 M_\odot/{\rm yr}$ lower bound. 
\end{itemize}

We demonstrate with our mock challenge that we will derive accurate and precise
constraints on galaxy properties in PROVABGS. 
Beyond $M_*$, $\avgsfr$, $\zmw$, and $\tage$, which we focus on in this work, 
PROVABGS will also constrain star formation and metallicity histories. 
With galaxy properties of over ${>}10$ BGS galaxies, current galaxy studies
will be able to use the PROVABGS catalog to exploit the statistical power of
BGS for the most precise measurements of various galaxy relations. 
Furthermore, since the BGS samples span a wide range of galaxies, PROVABGS will
also enable galaxy studies to investigate less explored regimes, such as the
low mass galaxy populations. 
PROVABGS will be a fully probabilistic catalog with posteriors for all the
properties that accurately capture their uncertainties.
With these posteriors, we can also conduct more rigorous statistical analyses
using new techinques such as population inference and Bayesian hierarchical
inference.
We use one such method, population inference, to estimate the overall accuracy
and precision of our galaxy property constraints. 
These methods will not only improve the accuracy of our analyses but will allow
us to fully exploit the statistical power of DESI observations. 

Despite the overall success of the PROVABGS methodologies, as demonstrated in
this mock challenge, there are some limitations. 
For instance, we only consider a simple model for the effect of the DESI fiber
aperture and flux calibration. 
We reserve a more detailed investigation for Ramos~\etal~(in prep.). 
We also do not consider varying the isochrones, stellar library, or IMF. 
Instead, we will release multiple versions of PROVABGS with different sets of
assumptions. 
Lastly, we find that the most significant limitation to deriving accurate
galaxy properties comes from the prior imposed by the SED model. 
We will address this limitation and present a method to impose uniform priors
on galaxy properties in the upcoming Hahn~\etal~(in prep.). 

DESI has started its main 5 year operation. 
Already, as part of survey validation, DESI has collected \todo{X} spectra of
BGS galaxies that will be released in the Survey Validation Data Assembly
(SVDA). 
The SVDA release will also be accompanied by papers describing the data
reduction pipeline, redshift fitting algorithm, fiber assignment, survey
operation and simulations, visual inspection, and target selection for the
various tracers. 
Finally, using BGS observations in the SVDA, we will construct and release the
PROVABGS-SV catalog and present the stellar mass function measured from it in
the subsequent paper. 


\section*{Acknowledgements}
It's a pleasure to thank
    Justin Alsing, 
    Adam Carnall, 
    Kartheik Iyer, 
    Joel Leja, 
    Jenny Greene, 
    and 
    Peter Melchior
for valuable discussions and comments. 
This material is based upon work supported by the U.S. Department of Energy,
Office of Science, Office of High Energy Physics, under contract No.
DE-AC02-05CH11231.  This project used resources of the National Energy Research
Scientific Computing Center, a DOE Office of Science User Facility supported by
the Office of Science of the U.S.  Department of Energy under Contract No.
DE-AC02-05CH11231. 
CH is supported by the AI Accelerator program of the Schmidt Futures Foundation.
MS is supported by the European Union's  Horizon 2020 research and innovation
programme under the Maria Sk\l{}odowska-Curie (grant agreement No 754510), the
National Science Centre of Poland (grant UMO-2016/23/N/ST9/02963) and by the
Spanish Ministry of Science and Innovation through Juan de la Cierva-formacion
program (reference FJC2018-038792-I).
MM acknowledges support from the Ramon y Cajal fellowship (RYC2019-027670-I).

This research is supported by the Director, Office of Science, Office of High
Energy Physics of the U.S. Department of Energy under Contract No.
DE–AC02–05CH11231, and by the National Energy Research Scientific Computing
Center, a DOE Office of Science User Facility under the same contract;
additional support for DESI is provided by the U.S. National Science
Foundation, Division of Astronomical Sciences under Contract No. AST-0950945 to
the NSF’s National Optical-Infrared Astronomy Research Laboratory; the Science
and Technologies Facilities Council of the United Kingdom; the Gordon and Betty
Moore Foundation; the Heising-Simons Foundation; the French Alternative
Energies and Atomic Energy Commission (CEA); the National Council of Science
and Technology of Mexico; the Ministry of Economy of Spain, and by the DESI
Member Institutions.

The authors are honored to be permitted to conduct scientific research on
Iolkam Du’ag (Kitt Peak), a mountain with particular significance to the Tohono
O’odham Nation.


\appendix
\section{Non-negative Matrix Factorization Bases} \label{sec:nmf}
The basis vectors for the star-formation and metallicity histories are computed
using non-negative matrix factorisation (NMF) on a set of star formation and
metallicity histories in the Illustris 
simulation~\citep{vogelsberger2014, genel2014, nelson2015}.
Unlike PCA, NMF lends itself well to this task as it gives positive vectors,
which can each be straightforwardly interpreted physically as representing the
SFH of a composite stellar population. 
In the case of the ZHs, the advantage of NMF over PCA is less clear, but we
maintain the NMF scheme for simplicity. 

The SFHs and ZHs are computed from all stellar particles bound to subhalos
that host a galaxy with $M_* > 10^9 M_\odot$ at $z=0$, giving a sample of just
over 29,000 Illustris galaxies. 
For the SFHs, we take the distribution of stellar ages in 400 bins,
logarithmically distributed between 8.6 Myrs and 13.65 Gyrs, and compute the
stellar mass formed in each bin. For the ZHs, we take the mass-weighted
metallicity in each of the bins. 
Next, the vectors for the SFHs and ZHs are normalized independently ---
\emph{i.e.} we do not keep information of which ZH corresponds to each SFH.
Therefore we do not impose the mass-metallicity relation of the simulation onto
our basis vectors (see \citealt{thorne2021} for a parameterization that links
SFH with ZH throught he mass-metallicity relation). 
We take each set of simulated SFHs and ZHs as a reasonable representation of
possible SFHs and ZHs in the Universe. 
Prior to decomposition, each individual vector is smoothed on a scale of 400
Myr, which removes any information on smaller timescales. 
We decompose the set of SFHs into 4 independent components, and the set of ZHs
into 2 independent components. 
The resulting components are shown in the main text (Figure~\ref{fig:nmf}). 

\begin{figure}
\begin{center}
\includegraphics[width=0.85\textwidth]{figs/NFMrec_gal2000_A.pdf}
\includegraphics[width=0.85\textwidth]{figs/NFMrec_gal2000_B.pdf}
    \caption{
    The original and NMF-reconstructed SFHs (top left) and ZHs (top right) of a
    galaxy in the Illustris simulation. 
    The original SFH is shown after smoothing on a scale of 400 Myr. 
    We mark the contributions of each of the NMF components in the faded
    colored lines. 
    The middle and bottom panels compare the spectra obtained from integrating
    the original and reconstructed SFH and ZH. 
    In this case, the NMF basis offers a good reconstruction of the SFH and ZH,
    which results in a small residuals in the corresponding spectra.
    }\label{fig:nmf0}
\end{center}
\end{figure}

\begin{figure}
\begin{center}
\includegraphics[width=0.85\textwidth]{figs/NFMrec_gal9009_A.pdf}
\includegraphics[width=0.85\textwidth]{figs/NFMrec_gal9009_B.pdf}
    \caption{
    Same as Figure~\ref{fig:nmf0} but for another Illustris galaxy. 
    In this case, the NMF basis fails to reproduce a burst of star formation at
    recent times, leading directly to an underestimation of the luminosity,
    especially towards the bluer wavelengths.
    }\label{fig:nmf1}
\end{center}
\end{figure}

\begin{figure}
\begin{center}
\includegraphics[width=0.85\textwidth]{figs/ensemble_plots_4p2.jpg}
    \caption{
    Comparison of original versus NMF reconstructed total stellar mass formed,
    mass-weighted age, mass-weighted metallicity, and mass formed in the last
    200 Myr (from top to bottom). 
    On the left column we present the distribution of each quantity, and on the
    right column we show direct comparisons in scatter plots. 
    The orange dashed lines shows the one-to-one line.
    Total stellar mass is well recovered, as expected given its lack of
    sensitivity to smaller bursts. 
    Mass-weighted ages are poorly recovered at young and old ages, as a direct
    consequence of the lack of resolution of our basis.
    The mass-weighted metallicity is well recovered on the mean, though with a
    large scatter. 
    The mass formed in young stars is again affected by the lack of resolution
    of our basis. 
    In our SPS model, we include a stochastic burst component to account for
    this limitation (Section~\ref{sec:sps}). 
    }\label{fig:nmf2}
\end{center}
\end{figure}

Figures~\ref{fig:nmf0} and~\ref{fig:nmf1} show two examples of the NMF direct
reconstruction on two galaxies. 
The two galaxies are chosen as examples of a `fair' and a `poor' reconstruction.
In all cases the reconstructions can be improved by increasing the number of
components, and doing so effectively improves our ability to model shorter
timescale features in the SFH and ZHs.
In this work, we instead include a stochastic burst component in the SFH
(Section~\ref{sec:sps}). 

In Figure~\ref{fig:nmf2}, we present how NMF reconstruction projects onto
certain derived properties: total stellar mass formed, mass-weighted age,
mass-weighted metallicity and mass in young stars (mass formed in the last 200
Myr).
Besides the total stellar mass, the other derived properties are impacted by
the lack of short timescale features. 
Our stochastic burst component directly addresses this limitiation. 
Therefore, the NMF basis can be seen as a reasonable and minimal set to recover
the broad shape of the star-formation and metallicity histories, which is
complemented in our SPS model by the stochastic burst component.

\section{SPS Model Priors} \label{sec:model_priors}
\begin{figure}
\begin{center}
\includegraphics[width=0.5\textwidth]{figs/model_prior.pdf}
    \caption{
    Priors imposed by our SPS model on $\log M_*$, 
    $\log \overline{\rm SSFR}_{\rm 1 Gyr}$, and $\log Z_{\rm MW}$ at $z=0.1$. 
    }
\label{fig:model_prior}
\end{center}
\end{figure}

\section{Population Inference} \label{sec:hyper}
% restatement/high level explanation of what we do
We quantify the accuracy and precision of the inferred galaxy properties from
our SED modeling using population hyperparameters 
$\eta_{\Delta} = \{\mu_{\Delta_\theta}, \sigma_{\Delta_\theta}\}$ 
(Section~\ref{sec:results}). 
These hyperparameters describe the distribution of the difference between the
inferred and true parameters, $\Delta_{\theta}$, assuming that the distribution
has a Gaussian functional form (Eq.~\ref{eq:eta_gauss}). 
The $\eta_\Delta$ values we present in this work are MAP estimates of
$p(\eta_\Delta | \{ X_i \})$, the probability distribution of $\eta_\Delta$
given some galaxy population observations. 
They are inferred using population inference as described in the main text and
Eqs~\ref{eq:popinf} - \ref{eq:popinf2}.
Our approach for quantifying the accuracy and precision has a number of key
advantages over other methods. 
For instance, a naive way to quantify the accuracy and precision would be to
estimate the median and standard deviations of individual posteriors then
averaging them. 
This assumes that each individual posterior is close to a Gaussian. 
As we later demonstrate, this is an incorrect assumption that reduces the
posterior distrubtion to point estimates. 
Another approach would be to stack the posteriors by summing up all of the
individual posteriors. 
Neither of these approaches mathematically estimate the distribution we are
actually interested in estimating: $p(\eta_\Delta | \{ X_i \})$. 
Moreover, both approaches are biased (\citealt{malz2020} recently demonstrated
this in the context of combining photometric redshift posteriors). 

\begin{figure}
\begin{center}
\includegraphics[width=0.45\textwidth]{figs/etas_demo.pdf}
    \caption{
    The $\log M_*$ distribution described by the accuracy and precision
    hyperparameters for galaxies with $10.6 < \log M_* < 10.8$:
    $\mathcal{N}(10.7 + \mu_{\Delta_\theta}, \sigma_{\Delta_\theta})$ 
    (black dashed).
    The hyperparameters are MAP estimates of $p( \mu_{\Delta_\theta}, 
    \sigma_{\Delta_\theta} | \{ X_i \})$ derived from population inference
    (Eq.~\ref{eq:popinf}-\ref{eq:popinf2}). 
    We include individual $\log M_*$ posteriors of several galaxies with $M_*
    \sim 10^{10.7} M_\odot$ for comparison.   
    The individual posteriors have a wide variety of shapes, which can bias 
    naive estimates of their accuracy and precision. 
    The comparison illustrates that 
    $\mathcal{N}(\mu_{\Delta_\theta}, \sigma_{\Delta_\theta})$ provides a
    robust estimate of the overall accuracy and precision of the inferred
    posteriors. 
    }\label{fig:eta_demo}
\end{center}
\end{figure}
We illustrate the population inference approach in Figure~\ref{fig:eta_demo} 
where we present the distribution of $\log M_*$ described by the accuracy and
precision hyperparameters derived for galaxies with $10.6 < \log M_* < 10.8$: 
$\mathcal{N}(\mu_{\Delta_\theta} + 10.7, \sigma_{\Delta_\theta})$ (black
dashed). 
For comparison, we plot posteriors of $\log M_*$ for several individual
galaxies with $\log M_* \sim 10.7$.  
There is significant variation in the individual posteriors and many of them
are not well described by a Gaussian distribution. 
This variation is an expected consequence of noise in the observables and
MCMC sampling.  
We note that estimating the accuracy and precision by stacking the posteriors,
for instance, significantly underestimates the precision.  
Meanwhile, the accuracy and precision hyperparameters capture the overall
accuracy and precision of the individual posteriors. 


\bibliographystyle{mnras}
\bibliography{gqp_mc} 
\end{document}
