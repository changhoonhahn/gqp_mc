\section{Joint SED modeling of Photometry and Spectra} \label{sec:methods}
PROVABGS will provide inferred galaxy properties derived from joint SED
modeling of DESI photometry and spectra. 
For the SED modeling, we use a state-of-the-art stellar population synthesis
(SPS) model that uses a non-parametric SFH with a star-burst, a non-parametric
ZH that varies with time, and a flexible dust attenuation prescription. 

% describe SFH prescription
The form of the SFH is one of the most important factors in the accuracy of an
SPS model.
In general, the form of the SFH requires balancing between being flexible enough
to describe the wide range of SFHs in observations while not being too flexible
that it can describe any SFH at the expense of constraining power.  
If the model SFH is not flexible enough to describe actual SFHs of galaxies,
then unbiased galaxy properties cannot be inferred using the SPS model. 
For instance, most SPS models~(\emph{e.g.} CIGALE,~\citealt{serra2011};
BAGPIPES,~\citealt{carnall2017}) use parametric SFH such as the exponentially
declining $\tau$-model.
Such functional forms, however, produce biased estimates of galaxy properties
(\emph{e.g.} $M_*$ and SFR) when used to fit mock observations of simulated 
galaxies~\citep{simha2014, pacific2015, carnall2018}.
On the other hand, many non-parametric forms of the SFH are overly flexible
and allow unphysical SFHs~\citep{leja2019}, which unncessarily increases 
parameter degeneracies and discards constraining power. 

In our SPS model, we use a non-parametric SFH with two components: one based on
non-negative matrix factorization (NMF) basis functions and a starburst component.
For the first component, SFH is a linear combination of NMF SFH bases:
\begin{equation} \label{eq:nmf} 
    {\rm SFH}^{\rm NMF} (t, t_{\rm age}) = \sum\limits_{i=1}^{4} \beta_i
    \frac{s_i^{\rm SFH}(t)}{\int\limits_0^{t_{\rm age}} s_i^{\rm SFH}(t) \,
    {\rm d}t}. 
\end{equation} 
$\{s^{\rm SFH}_i\}$ are the NMF basis functions and $\{\beta_i\}$ are the
coefficients. 
The integral in the denominator normalizes the NMF basis functions to unity. 
We constrain $\sum_i \beta_i = 1$, so the total SFH of the component over the
age of the galaxy ($t_{\rm age})$ is normalized to unity.
$\{s^{\rm SFH}_i\}$ are from Tojeiro \etal~(in prep.) and derived from the
IllustrisTNG cosmological hydrodynamic simulation~\citep{nelson2018,
pillepich2018, springel2018}.
The SFHs of simulated galaxies IllustrisTNG are compiled, rebinnined, and smoothed
\todo{more details here}. 
Afterwards, we perform non-negative matrix
factorization~\citep{lee1999,cichocki2009, fevotte2011} on the smooth SFHs to
derive $\{s^{\rm SFH}_i\}$. 
We find that 4 components is sufficient to accurately reconstruct the SFHs
from IllustrisTNG. 
Assuming that the SFHs of IllustrisTNG galaxies resemble the SFHs of actual
observed galaxies, our NMF form provides a compact and flexible representation
of the SFHs. 

The NMF basis functions are derived from smooth SFHs, which means that it does
not include any stochasticity. 
However, observations and high resolution zoom-in hydrodyanmical simulations
both find significant stochasticity in galaxy SFHs~\citep{sparre2017,
caplar2019, hahn2019b, iyer2020}. 
To include this stochasticity in our SPS model, we include a starburst
component that consists of a SSP in the SFH. 
For the total SFH, we use
\begin{equation}
    {\rm SFH} (t, t_{\rm age}) = (1 - f_{\rm burst})~{\rm SFH}^{\rm NMF} (t,
    t_{\rm age}) + f_{\rm burst}~\delta_{\rm D}(t - t_{\rm burst}).
\end{equation}
$f_{\rm burst}$ is the fraction of total stellar mass formed during the
starburst; $t_{\rm burst}$ is the time at which the starburst occurs; 
$\delta_{\rm D}$ is the Dirac delta function.
In totatl we have 6 free parameters in our SFH: 4 NMF basis coefficients 
($\beta_i$), $f_{\rm burst}$, and $t_{\rm burst}$. 

% describe ZH 
Another key part of an SPS model is the ZH, or chemical enrichment history. 
Current SPS models mostly assume a ZH that does not vary over
time~\citep{carnall2017, leja2019}. 
Since galaxies in hydrodynamical simulations and observations do not have
constant metallicities throughout their history, this assumption can
significantly bias the inferred galaxy properties. 
Instead, for our ZH, we take a similar approach to the SFH and use NMF basis
functions:
\begin{equation}
    {\rm ZH}(t) = \sum\limits_{i=1}^2 \gamma_i s_i^{\rm ZH}(t).
\end{equation} 
$\{s_i^{\rm ZH}(t)\}$ are the ZH NMF basis functions and $\{\gamma_i\}$ are the
coefficients. 
$\{s_i^{\rm ZH}(t)\}$ are fit using the ZHs of simulated galaxies from IllustrisTNG. 
We use two NMF components, so our ZH prescription has 2 free parameters. 



\begin{figure}
\begin{center}
\includegraphics[width=0.8\textwidth]{figs/nmf_bases.pdf} 
    \label{fig:nmf}
    \caption{
        Non-negative matrix factorization basis functions for the SFH
        (left) and ZH (right). 
        These basis functions are derived from the SFHs and ZHs of simulated
        galaxies in the IllustrisTNG cosmological hydrodynamic simulations. 
    }
\end{center}
\end{figure}

% describe construction of unattenuated spectra 
% log-spaced lookback time for SFH for NMF portion + burst 


% describe dust attenuation 

%description of our speculator SED model \citep{alsing2019}, which is based on FSPS. We use Chabrier IMF \ch{do we need to justify htis?}. 

Calzetti dust attenuation curve

Rather than evaluating the SPS directly using FSPS we use {\sc Speculator}, an
emulator for FSPS. Brief explanation of the PCA neural net. motivate why we're
using speculator by reiterating \cite{alsing2019}s speed profiling. 

describe our MCMC sampling. James's work in convergence here. talk about how in
principle speculator can easily be used with HMC because you can get
derivatives with backpropagation. provide detailed profiling of SED fitting and
projections for full 10million galaxy BGS sample. 

% describe how we derive galaxy properties
\begin{equation}
    {\rm SFR}_{100{\rm Myr}} = \frac{\int\limits_{t_{\rm age} - 100{\rm
    Myr}}^{t_{\rm age}}{\rm SFH}(t)\,{\rm d}t}{100\,{\rm Myr}}
\end{equation} 

\begin{figure}
\begin{center}
    \includegraphics[width=0.95\textwidth]{figs/mcmc_posterior_demo.pdf}
    \label{fig:posterior}
    \caption{
        \emph{Top}: 
        Posterior probability distribution of our 12 SPS model parameters
        derived from joint SED modeling of the mock DESI photometry and
        spectra.
        The contours mark the 68 and 95\% percentiles of the distribution. 
        \emph{Bottom}: 
        Comparison of our SPS model evaluated at the best-fit parameters
        (orange) with the mock observations (black). 
        On the left panel, we compare the $g$, $r$, $z$ band magnitudes; on the
        right, we compare the spectroscopy.  
    }
\end{center}
\end{figure}

In Figure~\ref{fig:posterior} we present the posterior as well as the best-fit photometry and
spectroscopy using speculator + emcee. 

%%%%%%%%%%%%%%%%%%%%%%%%%%%%%%%%%%%%%%%%%%%
%% forecast table
%%%%%%%%%%%%%%%%%%%%%%%%%%%%%%%%%%%%%%%%%%%
%\begin{table}
%\caption{Table of the different parameterizations for the spectro-photometric fitting. The changes in each setup with respect to default configuration, CIGALE D, are bolted. } 
%\begin{center} 
%\begin{tabular}{ccccc} \toprule
%name & SFH & dust & IMF \\[3pt]
%\hline 
%iFSPS C & rita's basis & & \\
%Firefly  & rita's basis & & \\
%VESPA    & & & \\
%CIGALE A & \textbf{delayed SFH} & Charlot $\&$ Fall 2000, Draine et al. 2014 & Salpter \\
%CIGALE B & delayed SFH + additional burst & \textbf{Calzetti et al. 2000, Dale et al. 2014} & Salpter \\
%CIGALE C & delayed SFH + additional burst & Charlot $\&$ Fall 2000, Draine et al. 2014 & \textbf{Chabrier} \\
%CIGALE D & delayed SFH + additional burst & Charlot $\&$ Fall 2000, Draine et al. 2014 & Salpeter \\[3pt]
%\hline            
%\end{tabular} \label{tab:setups}
%\end{center}
%\end{table}
%%%%%%%%%%%%%%%%%%%%%%%%%%%%%%%%%%%%%%%%%%%
