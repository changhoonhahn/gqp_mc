\section{Joint SED modeling of Photometry and Spectra} \label{sec:methods}
PROVABGS will provide inferred galaxy properties derived from joint SED
modeling of DESI photometry and spectra. 
For our SED modeling we use a state-of-the-art stellar population synthesis
(SPS) model based on \fsps~that uses non-parmaetric SFH and ZH, a flexible dust
attenuation model, and accounts for uncertainties in the stellar evolution. 

We don't use tau model for SFH because it biases the physical parameters
inferred. \todo{citation}. Instead we use the
following SFH bases from Rita in prep. 
\begin{equation}
    {\rm SFH}(t, t_{\rm age}) = \sum\limits_{i=1}^{4} \beta_i^{\rm SFH}
    \frac{s_i^{\rm SFH}(t)}{\int\limits_0^{t_{\rm age}} s_i^{\rm SFH}(t) \,
    {\rm d}t}
\end{equation} 
Similarly for ZH, we use the following ZH bases 
\begin{equation}
    {\rm ZH}(t) = \sum\limits_{i=1}^2 \gamma_i^{\rm ZH} s_i^{\rm ZH}(t)
\end{equation} 

\todo{describe model prior} 


\begin{itemize}
    \item be explicit about what physical properties are inferred from our SED
        fitting: stellar masses, ages, metallicities, star-formation histories.
        Expalin how we derive SFRs from the SFHs (i.e. our choice of 100Myr) 
    \item brief description of other standard SED fitting methods we include for
        comparison: firefly, VESPA, CIGALE
\end{itemize} 


description of our speculator SED model \citep{alsing2019}, which is based on
FSPS. We use Chabrier IMF \ch{do we need to justify htis?}. 

\begin{equation}
    {\rm SFR}_{100{\rm Myr}} = \frac{\int\limits_{t_{\rm age} - 100{\rm
    Myr}}^{t_{\rm age}}{\rm SFH}(t)\,{\rm d}t}{100\,{\rm Myr}}
\end{equation} 
see Figure~\ref{fig:speculator}. 

Calzetti dust attenuation curve

\begin{figure}
\begin{center}
\includegraphics[width=0.8\textwidth]{figs/speculator.pdf} 
\caption{speculator bases}
\label{fig:speculator}
\end{center}
\end{figure}
Rather than evaluating the SPS directly using FSPS we use {\sc Speculator}, an
emulator for FSPS. Brief explanation of the PCA neural net. motivate why we're
using speculator by reiterating \cite{alsing2019}s speed profiling. 

describe our MCMC sampling. James's work in convergence here. talk about how in
principle speculator can easily be used with HMC because you can get
derivatives with backpropagation. provide detailed profiling of SED fitting and
projections for full 10million galaxy BGS sample. 

In Figure XXXX we present the posterior as well as the best-fit photometry and
spectroscopy using speculator + emcee. 

%%%%%%%%%%%%%%%%%%%%%%%%%%%%%%%%%%%%%%%%%%%
%% forecast table
%%%%%%%%%%%%%%%%%%%%%%%%%%%%%%%%%%%%%%%%%%%
%\begin{table}
%\caption{Table of the different parameterizations for the spectro-photometric fitting. The changes in each setup with respect to default configuration, CIGALE D, are bolted. } 
%\begin{center} 
%\begin{tabular}{ccccc} \toprule
%name & SFH & dust & IMF \\[3pt]
%\hline 
%iFSPS C & rita's basis & & \\
%Firefly  & rita's basis & & \\
%VESPA    & & & \\
%CIGALE A & \textbf{delayed SFH} & Charlot $\&$ Fall 2000, Draine et al. 2014 & Salpter \\
%CIGALE B & delayed SFH + additional burst & \textbf{Calzetti et al. 2000, Dale et al. 2014} & Salpter \\
%CIGALE C & delayed SFH + additional burst & Charlot $\&$ Fall 2000, Draine et al. 2014 & \textbf{Chabrier} \\
%CIGALE D & delayed SFH + additional burst & Charlot $\&$ Fall 2000, Draine et al. 2014 & Salpeter \\[3pt]
%\hline            
%\end{tabular} \label{tab:setups}
%\end{center}
%\end{table}
%%%%%%%%%%%%%%%%%%%%%%%%%%%%%%%%%%%%%%%%%%%
