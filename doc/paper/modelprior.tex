\section{SPS Model Priors} \label{sec:model_priors}
SED models impose undesireable non-uniform priors on galaxy properties that
significantly impact their ability to infer unbiased galaxy properties such as
$\avgsfr$, $\zmw$, and $\tage$. 
For the PROVABGS SED modeling we present in this work, model imposed priors
place a lower bound on $\avgsfr$, biases $\zmw$ for observations with low
spectral SNR, and places an upper bound of $\tage < 8$ Gyr.
Given their significant impact, we quantify and characterize the model imposed
prior in further detail below. 

% explanation of how the parameters are derived 
Model imposed priors are a consequence of the fact that many of the galaxy
properties of interest are not explicit parameters of the SPS model. 
Out of the properties we focus on in this work, only $M_*$ is a parameter in
our SPS model. 
$M_*$ determines the overall amplitude of the SED. 
Meanwhile, $\avgsfr$ is derived from integrating, over the range 
$t_{\rm age} - 1\,{\rm Gyr} < t < t_{\rm age}$, the SFH, which is itself a
derived quantity from the SED model parameters $\{\beta_1, \beta_2, \beta_3,
\beta_4 \}$, $f_{\rm burst}$, $t_{\rm burst}$, and $M_*$ (Eq.~\ref{eq:nmf}
and~\ref{eq:sfh}).
$\zmw$ is an even more complicatedly derived quantity that involves integrating
the product of the SFH and ZH, and hence depends on $\{\beta_1, \beta_2,
\beta_3, \beta_4 \}$, $f_{\rm burst}$, $t_{\rm burst}$, and $\{\gamma_1,
\gamma_2\}$ (Eq.~\ref{eq:prop_eqs}).
$\tage$ is similarly derived by integrating the SFH by age. 
All of these derived properties are further impacted by the fixed log-spaced
$\tlb$ binning (Section~\ref{sec:sps}) since the integrals are evaluated
discretely. 

\begin{figure}
\begin{center}
\includegraphics[width=0.75\textwidth]{figs/model_prior.pdf}
    \caption{
    Priors imposed by our SPS model on galaxy properties $\log M_*$, $\log
    \overline{\rm SSFR}_{\rm 1 Gyr}$, $\log Z_{\rm MW}$, and $\tage$ at
    $z=0.1$. 
    Out of the galaxy properites, only $\log M_*$ is a parameter in our SPS
    model.
    The other properties are derived from the model parameters. 
    Hence, even when we impose uninformative priors on the model parameters as
    in Table~\ref{tab:params}, we do not impose uniform priors on the galaxy
    properties. 
    In fact, for $\avgssfr$, $\zmw$, and $\tage$, our SPS model imposes
    significantly skewed distributions that explain the biases and bounds we
    find in the galaxy property posteriors.  
    All SPS models impose undesireable priors on galaxy properties. 
    By characterizing the prior distribution above, we provide a way to
    interpret the posteriors on galaxy properties for PROVABGS and disentangle
    the effect of the prior. 
    }\label{fig:model_prior}
\end{center}
\end{figure}
% present the figure and describe in detail how we derive the figure. 
We further illustrate and quantify the model imposed priors on the galaxy
properties in Figure~\ref{model_prior}. 
We present the probability distribution of the model imposed prior on 
$\log M_*$, $\log\avgssfr$, $\log\zmw$, and $\tage$ for galaxies at $z=0.1$.
The prior distribution is derived by first sampling SPS parameters from prior
specified in Table~\ref{tab:params}, $\theta'\sim p(\theta)$.
Then for each $\theta'$, we derive the galaxy properties using
Eq.~\ref{eq:prop_eqs}. 
We present $\log\avgssfr = \log(\avgsfr/M_*)$ instead of $\log\avgsfr$ to
remove the correlation with $M_*$.  
The contours mark the 68 and 95\% of the distribution. 
We note that the prior distribution depends on redshift since it determines
$\tage$.
The dependence is relatively small so we only show $z=0.1$ for simplicity. 

% characterize the prior distribution 
We confirm that the prior on $\log M_*$ is uniform as we specify in
Table~\ref{tab:params}. 
For the other parameters, however, the model imposed prior is \emph{not} a
uniform distribution. 
For $\avgssfr$, the prior is a skewed distribution that spans $-13.5 < \log
\avgssfr < -9$ dex with two modes.
The secondary peak near -9 dex is a consequence of the starburst component that
we include in the SFH. 
By definition $\avgssfr$ cannot exceed $10^{-9}\,yr^{-1}$. 
For $\log\zmw$ and $\tage$, the marginalized priors are also skewed
distributions that peak near -1.6 dex and 6 Gyr, respectively.
Furthermore, for $\tage$, the prior reveals the imprint of the log-spaced 
$\tlb$ bins (see $\tage$ versus $\log\avgssfr$ panel).
As we discuss in the main text, the shape of the model imposed prior on
$\avgssfr$, $\zmw$, and $\tage$ explains the limitations of the posteriors we
derive from our SED modeling: the lower bound on $\avgsfr$, biases in $\zmw$,
and the $\tage$ upper bound (Section~\ref{sec:discuss}).


\begin{figure}
\begin{center}
\includegraphics[width=0.4\textwidth]{figs/model_prior_sfh.pdf}
    \caption{
        Priors imposed by our SPS model on the specific-SFH (sSFH) for galaxies
        at $z=0.1$. 
        We represent 68, 95, 99.7\% of the sSFH distribution with the shaded
        regions (dark to light). 
        Our SPS model imposes a prior on sSFH that is asymmetric and peaked at 
        ${\sim}5\times10^{-10}{\rm yr}^{-1}$. 
        This means that for observations where the likelihood distribution is
        diffuse (\emph{e.g.} low SNR), the inferred SFH will be significantly
        skewed towards the peak of the distribution. 
        Overall, this prior will cause the inferred SFHs to be flatter over
        $\tlb$ and skew towards the peak as we see in the comparison between
        the inferred and true SFHs in Figure~\ref{fig:sfh_demo}. 
        Any analysis of SFHs based on SED modeling must take account the effect
        of model imposed priors. 
    }\label{fig:sfh_prior}
\end{center}
\end{figure}
In addition to the galaxy properties above, we also characterize the model
imposed prior on specific-SFH (sSFH) in Figure~\ref{fig:sfh_prior}. 
The sSFH is the SFH normalized by total stellar mass. 
The shaded regions represent 68, 95, 99.7\% of the SFH distribution (dark to
light). 
We show the prior for galaxies at $z=0.1$. 
Throughout the $\tlb$ range, the sSFH prior is asymmetric and peaks at
${\sim}5\times10^{-10}{\rm yr}^{-1}$. 
Since this prior is implicitly included, the SFH posterior will also be skewed
towards this sSFH peak depending on the relative amplitude and width of the
likelihood distribution. 
In other words, the inferred SFH will generally be flatter as a function of
$\tlb$ than the true SFH. 
We can see this effect in Figure~\ref{fig:sfh_demo}. 
For the star-forming galaxy with a relatively flat SFH at intermediate values,
the inferred SFH is in good agreement with the true SFH. 
However, for the quiescent galaxy, which has high SFRs at early times ($\tlb >
6$ Gyr) and low SFRs at late times ($\tlb < 2$ Gyr), the inferred SFH is
flatter and skewed towards intermediate values. 
We note that the prior on SFH is similar to the priors on SFHs by various
nonparametric SPS models in \cite{leja2019}. 
Hence, any detailed analysis of SFHs (\emph{e.g.} quenching timescale or star
formation variable) based on SED modeling must take the impact of model imposed
priors on SFH into account or taken with a grain of salt.  

We emphasize that all SPS models impose undesirable priors on derived galaxy
properties. 
And \emph{any} deviation of the priors on galaxy properties from a uniform
distribution impacts the posteriors on the galaxy properties. 
Galaxy properties derived from any SED modeling must, therefore, characterize
and account for the priors imposed on them by the model for unbiased and
accurate analyses. 
In this appendix, we characterize the model imposed priors of our PROVABGS SED
model for the main galaxy properties that we explore in this work. 
This allows us to interpret the posteriors of galaxy properties for PROVABGS
and qualitatively disentangle the effect of the prior. 
In an accompanying work (Hahn in prep.), I will demonstrate that
maximum-entropy priors can be used to substantially mitigate the impact of
model impose priors on the posteriors of galaxy properties
(see Section~\ref{sec:discuss}). 
