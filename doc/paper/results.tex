\section{Results} \label{sec:results}
%\subsection{Inferred Galaxy Properties}
Figuree~\ref{fig:prop_inf} 

\begin{figure}
\begin{center}
\includegraphics[width=\textwidth]{figs/mini_mocha_sfr_100myr_comparison.pdf} 
\caption{The properties inferred from ifsps spetrophotometry fit as a function of true properties. 
}
\label{fig:prop_inf}
\end{center}
\end{figure}

In order to quantify the precision and accuracy of the inferred physical
properties for our simulated galaxy population, we begin by assuming that the
discrepancy between the inferred and true parameters for each galaxy 
$\Delta_{\theta,i}$) 
\begin{equation}
    \theta^{\rm inf}_i = \theta^{\rm true}_i + \Delta_{\theta,i}
\end{equation}
where $\Delta_{\theta,i}$ is sampled from a Gaussian distribution
\begin{equation}
    \Delta_{\theta,i} \sim \mathcal{N}(\mu_{\Delta_{\theta}}, \sigma_{\Delta_{\theta}}).
\end{equation}
This Gaussian distribution is described by population hyperparameters $\mu_{\Delta_{\theta}}$ and 
$\sigma_{\Delta_{\theta}}$, the mean and standard deviation, which quantify the accuracy and 
precision of the inferred physical properties for the population. 

Given the photomety and spectrum of our galaxies, $\{{\bfi D}_i\}$, we can get the posteriors
for these population parameters $\theta_\Delta = (\mu_{\Delta_{\theta}}, \sigma_{\Delta_{\theta}})$ 
using a hierarchical Bayesian framework~\citep{hogg2010a}: 
\begin{align}
p(\theta_\Delta \given \{{\bfi D_i}\}) 
    =&~\frac{p(\theta_\Delta)~p( \{{\bfi D_i}\} \given \theta_{\Delta})}{p(\{{\bfi D_i}\})}\\
    =&~\frac{p(\theta_\Delta)}{p(\{{\bfi D_i}\})}\int p(\{{\bfi D_i}\} \given \{\theta_i\})~p(\{\theta_i\} \given \theta_\Delta)~{\rm d}\{\theta_i\}.
\end{align} 
Naively the posteriors for each of the galaxies are not correlated, so we can factorize the expression above
\begin{align}
p(\theta_\Delta \given \{{\bfi D_i}\}) 
    =&~\frac{p(\theta_\Delta)}{p(\{{\bfi D_i}\})}\prod\limits_{i=1}^N\int p({\bfi D_i} \given \theta_i)~p(\theta_i \given \theta_\Delta)~{\rm d}\theta_i\\
    =&~\frac{p(\theta_\Delta)}{p(\{{\bfi D_i}\})}\prod\limits_{i=1}^N\int \frac{p(\theta_i \given {\bfi D_i})~p({\bfi D_i})}{p(\theta_i)}~p(\theta_i \given \theta_\Delta)~{\rm d}\theta_i\\
    =&~p(\theta_\Delta)\prod\limits_{i=1}^N\int \frac{p(\theta_i \given {\bfi D_i})~p(\theta_i \given \theta_\Delta)}{p(\theta_i)}~{\rm d}\theta_i.
\end{align} 
$p(\theta_i \given {\bfi D_i})$ is the posterior for galaxy $i$. Hence, the
integral can be which means the integral can be estimated using the MCMC sample
from the posterior
\begin{align}
p(\theta_\Delta \given \{{\bfi D_i}\}) 
    =&~p(\theta_\Delta)\prod\limits_{i=1}^N\frac{1}{S_i}\sum\limits_{j=1}^{S_i}
    \frac{p(\theta_{i,j} \given \theta_\Delta)}{p(\theta_{i,j})}.
\end{align} 
$S_i$ is the number of MCMC samples and $\theta_{i,j}$ is the $j^{\rm th}$
sample of galaxy $i$. We present the maximum a posteriori (MAP) estimates of
$\theta_\Delta$ for $\log~M_*$ and $\log~{\rm SFR}$ in
Figure~\ref{fig:specphoto}. 


$\theta_\Delta$ as a function of SNR/mag/colour. 

$\theta_\Delta$ as a function of SNR/mag/colour. 

\begin{figure}
\begin{center}
\includegraphics[width=0.75\textwidth]{figs/photo_vs_specphoto_ifsps_sfr_100myr_vanilla_noise_legacy_bgs0_legacy.pdf} 
\caption{The discrepancies between the inferred and input/``true'' $M_*$s (left) and SFRs 
(right) for our {\sc LGal} galaxies. In blue, we infer $M_*$s and SFRs using only photometry;
in orange, we infer $M_*$s and SFRs by jointly fitting both photometry and spectroscopy. 
{\em Jointly fitting spectroscopy and photometry improves constraints on galaxy properties.}
}
\label{fig:specphoto}
\end{center}
\end{figure}

%\begin{table}
%\caption{$\theta_\Delta$ $\theta_{\rm inf}$ - $\theta_{\rm true}$ and uncertainties for different sets of data fitted with ifsps} 
%\begin{center} 
%\begin{tabular}{ccccc} \toprule
%set & photometry & spectroscopy & specphot \\
%$\Delta M_{tot}$ & 0.13 & 0.11 & 0.09\\
%$M_{err}$ & 0.10 & 0.08 & 0.07 \\
%$\Delta$ Age & 4.05 & 3.84 & 4.33\\
%$Age_{err}$ & 1.83 & 2.37 & 2.03\\
%$\Delta$ Z & 0.0546 & 0.0126 & 0.0063 \\
%$Z_{err}$ & 0.0291 & 0.0203 & 0.0050\\
%\hline 
%\hline            
%\end{tabular} \label{tab:setups}
%\end{center}
%\end{table}

%\begin{figure}
%\begin{center}
%\includegraphics[width=\textwidth]{figs/mini_mocha_cigale_noise_CIGALEA.png} 
%\includegraphics[width=\textwidth]{figs/mini_mocha_cigale_noise_CIGALEB.png} 
%\includegraphics[width=\textwidth]{figs/mini_mocha_cigale_noise_CIGALEC.png}
%\includegraphics[width=\textwidth]{figs/mini_mocha_cigale_noise_CIGALED.png}
%\caption{The properties inferred from CIGALE photometry fit as a function of true properties. Configuration CIGALE A, B, C, and D 
%}
%\label{fig:photo_cigale}
%\end{center}
%\end{figure}

%\begin{figure}
%\begin{center}
%\includegraphics[width=\textwidth]{figs/mini_mocha_ifsps_specphotofit_vanilla_noise_bgs0_legacy_delta.pdf} 
%\caption{delta(galaxy properties) as a function of $M_{tot}$, r mag and colors for ifsps(spectrophotometry) and CIGALE (photometry, CIGALE D). 
%}
%\label{fig:photo_cigaleALL}
%\end{center}
%\end{figure}

{\bf figure}: $\theta_{\rm inf}$ vs $\theta_{\rm true}$ plot for final fitting methods. Money plot of the mock challenge 

{\bf figure}: delta(galaxy properties) as a function of r-mag, SNR(?), color for different fitters for spectro-photometric fitting 

{\bf figure}: $\theta_{\rm inf}$ vs $\theta_{\rm true}$ plot for multiple fitting methods 

